\documentclass{report}

\usepackage{ctex}

\input{../module/preamble}
\input{../module/macros}
\input{../module/letterfonts}

\ctexset{
    today=old
}

\title{\Huge{Linear Algebra}\\ Assignment 2}
\author{\Large{110307039\ 財管四\ 黃柏維}}
\date{\today}

\begin{document}

\maketitle
\cleardoublepage

\qs{}{(Page 92, Q26) Determine whether the stochastic matrix $P$ is regular. Then
    find the steady state matrix \( \overline{X} \) of the Markov chain with matrix
    of the transition probabilities $P$, where
    \[
        P = \begin{bmatrix}
            \frac{1}{2} & \frac{1}{5} & 1 \\
            \frac{1}{3} & \frac{1}{5} & 0 \\
            \frac{1}{6} & \frac{3}{5} & 0
        \end{bmatrix}.
    \]
}
\solve{
    \begin{enumerate}
        \item To determine whether the stochastic matrix $P$ is regular, we need to check
              whether the matrix $P^n$ has all positive elements for some $n$. We have
              \[
                  P^2 = \begin{bmatrix}
                      \frac{29}{60} & \frac{37}{50}  & \frac{1}{2} \\
                      \frac{7}{30}  & \frac{8}{75}\, & \frac{1}{3} \\
                      \frac{17}{60} & \frac{23}{150} & \frac{1}{6}
                  \end{bmatrix}
              \]
              Since all elements of $P^2$ are positive, the stochastic matrix $P$ is regular.

        \item To find the steady state matrix $\overline{X} \triangleq \begin{bmatrix}
                      x_1 \\ x_2 \\ x_3
                  \end{bmatrix}$ of the Markov chain with matrix of the transition probabilities $P$, we need to solve the equation
              \[
                  P\overline{X} = \overline{X}
              \]
              which is equivalent to
              \[
                  (P - I)\overline{X} = \mathbf{0}
              \]
              where $I$ is the identity matrix. We have
              \[
                  \overline{X} \in \mathrm{ker}(P - I) = \mathrm{span}\left\{ \begin{bmatrix}
                      12 \\ 5 \\ 5
                  \end{bmatrix} \right\},
              \]
              and by \( x_1 + x_2 + x_3 = 1\), we have
              \[
                  \overline{X} = \begin{bmatrix}
                      \frac{12}{22} \\ \frac{5}{22} \\ \frac{5}{22}
                  \end{bmatrix}
              \]
    \end{enumerate}
}

\qs{}{(Page 92, Q28) Determine whether the stochastic matrix $P$ is regular. Then
    find the steady state matrix \( \overline{X} \) of the Markov chain with matrix
    of the transition probabilities $P$, where
    \[
        P = \begin{bmatrix}
            0.1 & 0 & 0.3 \\
            0.7 & 1 & 0.3 \\
            0.2 & 0 & 0.4
        \end{bmatrix}.
    \]
}
\solve{
To determine whether the stochastic matrix $P$ is regular, we need to check
whether the matrix $P^n$ has all positive elements for some $n$. We have
\[
    P^n_{12} = P^n_{32} = 0, \forall n > 1
\]
Therefore, the stochastic matrix $P$ is not regular, and the steady state
matrix $\overline{X}$ does not exist. }

\qs{}{(Page 94, Q42) Find the steady state matrix \( \overline{X} \) of the
    absorbing Markov chain with matrix of transition probabilities \(P\), where
    \[
        P = \begin{bmatrix}
            0.1 & 0 & 0 \\
            0.2 & 1 & 0 \\
            0.7 & 0 & 1
        \end{bmatrix}.
    \]
}
\solve{
    Use the matrix equation \( P\overline{X} = \overline{X}\), along with the
    equation \( x_1 + x_2 + x_3 + x_4 = 1 \) to write the system of linear
    equations
    \[
        \begin{cases}
            0.1x_1 = x_1       \\
            0.2x_1 + x_2 = x_2 \\
            0.7x_1 + x_3 = x_3
        \end{cases}.
    \]
    The solution of this system is \( x_1 = 0, x_2 = 1 - t, x_3 = t \), where \( t
    \in \mathbb{R} \), and \( 0 \leq t \leq 1 \). \\ So, the steady state matrix is \[
        \overline{X} = \begin{bmatrix}
            0 \\ 1 - t \\ t
        \end{bmatrix}
    \]

}

\qs{}{(Page 102, Q10) A code braeker intercepted the encoded message below.
    \[
        45\ - 35\ 38\ - 30\ 18\ - 18\ 35\ - 30\ 81\ - 60\ 42\ - 28\ 75\ - 55\ 2\ - 2\ 22\ - 21\ 15\ - 10
    \]
    Let the inverse of the encoding matrix be
    \[
        A^{-1} = \begin{bmatrix}
            w & x \\
            y & z
        \end{bmatrix}
    \]
    \begin{enumerate}[label=(\alph*)]
        \item You know that the \( \begin{bmatrix}
                  45 & -35
              \end{bmatrix} A^{-1} = \begin{bmatrix}
                  10 & 15
              \end{bmatrix} \) and \( \begin{bmatrix}
                  38 & -30
              \end{bmatrix} A^{-1} = \begin{bmatrix}
                  8 & 14
              \end{bmatrix} \). Write and solve two systems of equations to find \( w, x, y, \text{ and } z \).

        \item Decode the message.
    \end{enumerate}
}
\solve{
    \begin{enumerate}[label=(\alph*)]
        \item By the given information, we have
              \[
                  \begin{cases}
                      45w - 35y = 10 \\
                      45x - 35z = 15
                  \end{cases}
              \]
              and
              \[
                  \begin{cases}
                      38w - 30y = 8 \\
                      38x - 30z = 14
                  \end{cases}
              \]
              Solving these two systems of equations, we get \( w = 1, x = - 2, y = 1, z = -
              3 \).

        \item Decoding the message by putting the encoded message into the matrix form, we
              have
              \[
                  \begin{bmatrix}
                      45 & - 35 \\
                      38 & - 30 \\
                      18 & - 18 \\
                      35 & - 30 \\
                      81 & - 60 \\
                      42 & - 28 \\
                      75 & - 55 \\
                      2  & - 2  \\
                      22 & - 21 \\
                      15 & - 10
                  \end{bmatrix} \begin{bmatrix}
                      1 & -2 \\
                      1 & -3
                  \end{bmatrix} = \begin{bmatrix}
                      10   & 15 \\
                      8    & 14 \\
                      0    & 18 \\
                      5    & 20 \\
                      21\,  & 18 \\
                      14   & 0  \\
                      20\, & 15 \\
                      0    & 2  \\
                      1\: & 19 \\
                      5    & 0
                  \end{bmatrix}
              \]
              The decoded message is \( 10\ 15\ 8\ 14\ 0\ 18\ 5\ 20\ 21\ 18\ 14\ 0\ 20\ 15\
              0\ 2\ 1\ 19\ 5\ 0 \). \\ which means "\( \textbf{JOHN RETURN TO BASE } \)" be
              the code table at page 94.
    \end{enumerate}
}

\qs{}{(Page 103, Q26) Find the least squares regression lline for the data
    \[
        (0, 6), (4, 3), (5, 0), (8, -4), (10, - 5)
    \]
}
\solve{
    The matrices \( X \) and \( Y \) are
    \[
        X = \begin{bmatrix}
            1 & 0  \\
            1 & 4  \\
            1 & 5  \\
            1 & 8  \\
            1 & 10
        \end{bmatrix}, \quad Y = \begin{bmatrix}
            6  \\
            3  \\
            0  \\
            -4 \\
            -5
        \end{bmatrix}
    \]
    Then, by the normal equation \( X^TXA = X^TY \), we have
    \[
        \begin{bmatrix}
            5  & 27  \\
            27 & 205
        \end{bmatrix} A
        \end{bmatrix} = \begin{bmatrix}
            0 \\
            - 70
        \end{bmatrix}
    \]
    Solving this system of equations, we get
    \[
        A = \begin{bmatrix}
            \frac{945}{148} \\
            - \frac{175}{148}
        \end{bmatrix}
    \]
    Therefore, the least squares regression line is
    \[
        y = \frac{945}{148} - \frac{175}{148}x.
    \]
}

\qs{}{(Page 116, Q24) Use expansion by cofactors to find the determinant of the matrix \( A \),
    where
    \[
        A = \begin{bmatrix}
            0.1   & 0.2 & 0.3 \\
            - 0.3 & 0.2 & 0.2 \\
            0.5   & 0.4 & 0.4
        \end{bmatrix}.
    \]
}
\solve{
    The cofactor expansion by the first row of \( A \) is
    \[
        \det(A) = 0.1 \begin{vmatrix}
            0.2 & 0.2 \\
            0.4 & 0.4
        \end{vmatrix} - 0.2 \begin{vmatrix}
            - 0.3 & 0.2 \\
            0.5   & 0.4
        \end{vmatrix} + 0.3 \begin{vmatrix}
            - 0.3 & 0.2 \\
            0.5   & 0.4
        \end{vmatrix}
    \]
    Solving these determinants, we get
    \[
        \det(A) = 0.1 \cdot 0 - 0.2(- 0.22) + 0.3(- 0.22) = 0 + 0.044 - 0.066 = - 0.022
    \]
}

\qs{}{(Page 116, Q28) Use expansion by cofactors to find the determinant of the matrix \( A \),
    where
    \[
        A = \begin{bmatrix}
            3 & 0 & 7 & 0 \\
            2 & 6 & 11 & 12 \\
            4 & 1 & -1 & 2 \\
            1 & 5 & 2 & 10
        \end{bmatrix}.
    \]
}
\solve{
    The cofactor expansion by the first row of \( A \) is
    \[
    \begin{align}
        \det(A) &= 3 \begin{vmatrix}
            6  & 11 & 12 \\
            1  & -1 & 2  \\
            5  & 2  & 10
        \end{vmatrix} - 0 + 7 \begin{vmatrix}
            2  & 11 & 12 \\
            4  & -1 & 2  \\
            1  & 2  & 10
        \end{vmatrix} - 0
        &= 3 \cdot 0 - 7 \cdot (- 338) = 2366.
    \end{align}
    \]
    Note that the first term is zero because the first column is linearly dependent on the third column.    
}    
\end{document}