\documentclass{report}

\usepackage{ctex}

\input{../module/preamble}
\input{../module/macros}
\input{../module/letterfonts}

\ctexset{
    today=old
}

\title{\Huge{Linear Algebra}\\ Assignment 4}
\author{\Large{110307039\ 財管四\ 黃柏維}}
\date{\today}

\begin{document}

\maketitle
\cleardoublepage

\qs{}{(CH4.4 P184 Q6) For the matrices \[
        A = \begin{bmatrix}
            2 & - 3 \\
            4 & 1   \\
        \end{bmatrix} \text{ and } B = \begin{bmatrix}
            0 & 5   \\
            1 & - 2 \\
        \end{bmatrix}
    \]
    in \( M_{2 \times 2} \), determine whether matrix \[
        C = \begin{bmatrix}
            6 & 2  \\
            9 & 11 \\
        \end{bmatrix}
    \]
    is a linear combination of \( A \) and \( B \). }

\solve{ Let \( C = xA + yB \)
    for some \( x, y \in \mathbb{R} \). Then we have the system of equations:
    \[
        \begin{align*}
                     & x \begin{bmatrix}
                             2 & -3 \\
                             4 & 1  \\
                         \end{bmatrix} + y \begin{bmatrix}
                                               0 & 5  \\
                                               1 & -2 \\
                                           \end{bmatrix} = \begin{bmatrix}
                                                               6 & 2  \\
                                                               9 & 11 \\
                                                           \end{bmatrix} \\[0.5em]
            \implies & \begin{cases*}
                           2x + 0y = 6  \\
                           -3x + 5y = 2 \\
                           4x + 1y = 9  \\
                           1x - 2y = 11 \\
                       \end{cases*}                                      \\
        \end{align*}
    \]
    Because the above system of equations are not consistent, \( C \) is not a
    linear combination of \( A \) and \( B \). }

\qs{}{(CH4.4 P184 Q18) Determine whether the set \[ S = \{ ( - 1, 2), (2, - 1), (1, 1) \} \] spans \( \RR[2] \). If the set does not span \( \RR[2] \), then give a
    geometric description of the subspace that it does span. }

\solve{ We first
    form the matrix \( A \) whose rows are the vectors in \( S \):
    \[
        A = \begin{bmatrix}
            -1 & 2  \\
            2  & -1 \\
            1  & 1  \\
        \end{bmatrix},
    \]
    which can be row reduced to
    \[
        \begin{bmatrix}
            1 & 0 \\
            0 & 1 \\
            0 & 0 \\
        \end{bmatrix} \sim A.
    \]
    Since the rank of \( A \) is 2, which is equal to the dimension of \( \RR[2]
    \), we conclude that \( S \) spans \( \RR[2] \). }

\qs{}{(CH4.4 P184 Q26) Determine whether the set \[ S = \{ -2x + x^2, 8x + x^3, - x^2 + x^3, -4 + x^2 \} \] spans \( P_3 \). }

\solve{ We first form the matrix \( A \) whose rows are the
    coefficients of the vectors in \( S \):
    \[
        A = \begin{bmatrix}
            -2 & 0  & 1 & 0 \\
            8  & 1  & 0 & 0 \\
            0  & -1 & 1 & 0 \\
            -4 & 0  & 1 & 0 \\
        \end{bmatrix},
    \]
    which can be row reduced to
    \[
        \begin{bmatrix}
            1 & 0 & 0 & 0 \\
            0 & 1 & 0 & 0 \\
            0 & 0 & 1 & 0 \\
            0 & 0 & 0 & 0 \\
        \end{bmatrix} \sim A.
    \]
    \( \because \mathrm{rank}(A) = 3 < \mathrm{rank}(P_3) = 4 \) \\
    \( \therefore S \) does not span \( P_3 \). \\
}

\qs{}{(CH4.5 P194 Q56) Determine whether \[ S = \{ \left( \frac{2}{3}, \frac{5}{2}, 1 \right), \left( 1, \frac{3}{2}, 0 \right), (2, 12, 6) \} \]
    is a basis for \( \RR[3] \). If it is, write \( \bmu = (8,3,8) \) as a linear
    combination of the vectors in \( S \). }

\solve{ We first form the matrix \( A
    \) whose rows are the vectors in \( S \):
    \[
        A = \begin{bmatrix}
            \frac{2}{3} & \frac{5}{2} & 1 \\
            1           & \frac{3}{2} & 0 \\
            2           & 12          & 6 \\
        \end{bmatrix},
    \]
    which can be row reduced to
    \[
        \begin{bmatrix}
            1 & 0 & - 1         \\
            0 & 1 & \frac{2}{3} \\
            0 & 0 & 0           \\
        \end{bmatrix} \sim A.
    \]
    \( \because \mathrm{rank}(A) = 2 < \mathrm{rank}(\RR[3]) = 3 \) \\
    \( \therefore S \) is not a basis for \( \RR[3] \). \\
}

\qs{}{(CH4.5 P194 Q68) Find all subsets of the set \[ S = \{ (1, 3, - 2), ( - 4, 1, 1), ( - 2, 7, - 3), (2, 1, 1) \} \] that form a bsis for \( \RR[3] \). }

\solve{ Let the vectors in \( S \) be \(
    \bmv_1, \bmv_2, \bmv_3, \bmv_4 \) in order. \\ Then we can test whether the
    matrices \( A_1, A_2, A_3, A_4 \) are rank 3, where \( A_i \) is the matrix
    whose rows are the vectors in \( S \) excluding \( \bmv_i \):
    \[
        \begin{align*}
            A_1 & = \begin{bmatrix}
                        -4 & 1 & 1  \\
                        -2 & 7 & -3 \\
                        2  & 1 & 1  \\
                    \end{bmatrix} \sim \begin{bmatrix}
                                           1 & 0 & 0 \\
                                           0 & 1 & 0 \\
                                           0 & 0 & 1 \\
                                       \end{bmatrix},         \\[0.5em]
            A_2 & = \begin{bmatrix}
                        1  & 3 & -2 \\
                        -2 & 7 & -3 \\
                        2  & 1 & 1  \\
                    \end{bmatrix} \sim \begin{bmatrix}
                                           1 & 0 & 0 \\
                                           0 & 1 & 0 \\
                                           0 & 0 & 1 \\
                                       \end{bmatrix},         \\[0.5em]
            A_3 & = \begin{bmatrix}
                        1  & 3 & -2 \\
                        -4 & 1 & 1  \\
                        2  & 1 & 1  \\
                    \end{bmatrix} \sim \begin{bmatrix}
                                           1 & 0 & 0 \\
                                           0 & 1 & 0 \\
                                           0 & 0 & 1 \\
                                       \end{bmatrix},         \\[0.5em]
            A_4 & = \begin{bmatrix}
                        1  & 3 & -2 \\
                        -4 & 1 & 1  \\
                        -2 & 7 & -3 \\
                    \end{bmatrix} \sim \begin{bmatrix}
                                           1 & 0 & - \frac{5}{13} \\
                                           0 & 1 & - \frac{7}{13} \\
                                           0 & 0 & 0              \\
                                       \end{bmatrix}.
        \end{align*}
    \]
    \( \because \mathrm{rank}(A_1) = \mathrm{rank}(A_2) = \mathrm{rank}(A_3) = 3 = \mathrm{rank}(\RR[3]) \) and \( \mathrm{rank}(A_4) = 2 < \mathrm{rank}(\RR[3]) = 3 \) \\
    \( \therefore \) The subsets \( \{ \bmv_2, \bmv_3, \bmv_4 \}, \{ \bmv_1, \bmv_3, \bmv_4 \}, \{ \bmv_1, \bmv_2, \bmv_4 \} \) form a basis for \( \RR[3] \). \\
}

\qs{}{(CH4.5 P194 Q70) Find a basis for \( \RR[3] \) that includes the vectors \( (1, 0, 2) \text{ and } (0, 1, 1) \). }

\solve{
    We can directly try a standard vector \( \bme_3 = (0, 0, 1) \).
    \[
        A \triangleq \begin{bmatrix}
            1 & 0 & 2 \\
            0 & 1 & 1 \\
            0 & 0 & 1 \\
        \end{bmatrix} \sim \begin{bmatrix}
            1 & 0 & 0 \\
            0 & 1 & 0 \\
            0 & 0 & 1 \\
        \end{bmatrix}.
    \]
    \( \therefore S = \{ (1, 0, 2), (0, 1, 1), (0, 0, 1) \} \) is a basis for \( \RR[3] \). \\
}

\qs{}{(CH4.6 P205 Q14) Find a basis for the subspace of \( \RR[3] \) spanned by \[
        S = \{ (2, 3, - 1), (1, 3, - 9), (0, 1, 5) \}.
    \]
}

\solve{
    We first form the matrix \( A \) whose rows are the vectors in \( S \):
    \[
        A = \begin{bmatrix}
            2 & 3 & -1 \\
            1 & 3 & -9 \\
            0 & 1 & 5  \\
        \end{bmatrix},
    \]
    which can be row reduced to
    \[
        \begin{bmatrix}
            1 & 0 & 0 \\
            0 & 1 & 0 \\
            0 & 0 & 1 \\
        \end{bmatrix} \sim A.
    \]
    \( \therefore S \) is a basis for the subspace of \( \RR[3] \) spanned by \( S \). \\
}

\qs{}{(CH4.6 P205 Q20) Find a basis for the subspace of \( \RR[4] \) spanned by \[
        S = \{ (2, 4, - 3, - 2), ( - 2, - 3, 2, - 5), (1, 3, - 2, 2), ( - 1, - 5, 3, 5) \}.
    \]}

\solve{
    We first form the matrix \( A \) whose rows are the vectors in \( S \):
    \[
        A = \begin{bmatrix}
            2  & 4  & -3 & -2 \\
            -2 & -3 & 2  & -5 \\
            1  & 3  & -2 & 2  \\
            -1 & -5 & 3  & 5  \\
        \end{bmatrix},
    \]
    which can be row reduced to
    \[
        \begin{bmatrix}
            1 & 0 & 0 & 3    \\
            0 & 1 & 0 & - 13 \\
            0 & 0 & 1 & - 19 \\
            0 & 0 & 0 & 0    \\
        \end{bmatrix} \sim A.
    \]
    \( \therefore \{ (2, 4, - 3, - 2), ( - 2, - 3, 2, - 5), (1, 3, - 2, 2) \} \) is a basis for the subspace of \( \RR[4] \) spanned by \( S \). \\
}

\qs{}{(CH4.6 P205 Q38) Find the nullspace of the matrix \[
        A = \begin{bmatrix}
            1   & 4   & 2   & 1   \\
            0   & 1   & 1   & - 1 \\
            - 2 & - 8 & - 4 & - 2 \\
        \end{bmatrix}.
    \]}

\solve{
    \[
        A = \begin{bmatrix}
            1   & 4   & 2   & 1   \\
            0   & 1   & 1   & - 1 \\
            - 2 & - 8 & - 4 & - 2 \\
        \end{bmatrix} \sim \begin{bmatrix}
            1 & 0 & - 2 & 5   \\
            0 & 1 & 1   & - 1 \\
            0 & 0 & 0   & 0   \\
        \end{bmatrix}.
    \]
    The nullspace of \( A \) is therefore spanned by \( S \), where \[ S = \left\{
        \begin{bmatrix}
            2  \\
            -1 \\
            1  \\
            0  \\
        \end{bmatrix},
        \begin{bmatrix}
            -5 \\
            1  \\
            0  \\
            1  \\
        \end{bmatrix}
        \right\}.
    \]
}

\qs{}{(CH4.6 P205 Q40) Find the nullspace of the matrix \[
        A = \begin{bmatrix}
            1 & 4   & 2 & 1 \\
            2 & - 1 & 1 & 1 \\
            4 & 2   & 1 & 1 \\
            0 & 4   & 2 & 0
        \end{bmatrix}.
    \]}

\solve{
    \[
        A = \begin{bmatrix}
            1 & 4   & 2 & 1 \\
            2 & - 1 & 1 & 1 \\
            4 & 2   & 1 & 1 \\
            0 & 4   & 2 & 0
        \end{bmatrix} \sim \begin{bmatrix}
            1 & 0 & 0 & 0 \\
            0 & 1 & 0 & 0 \\
            0 & 0 & 1 & 0 \\
            0 & 0 & 0 & 1 \\
        \end{bmatrix}.
    \]
    The nullspace of \( A \) is therefore spanned by \( S \), where \[ S = \left\{
        \begin{bmatrix}
            0 \\
            0 \\
            0 \\
            0 \\
        \end{bmatrix}
        \right\}.
    \]
}

\qs{}{(CH4.6 P206 Q42) Use the fact that matrices \( A \) and \( B \) are row-equivalent to \begin{enumerate}[label=(\alph*)]
        \item Find the rank and nullity of \( A \).
        \item Find a basis for the nullspace of \( A \).
        \item Find a basis for the row space of \( A \).
        \item Find a basis for the column space of \( A \).
        \item Determine whether the rows of \( A \) are linearly independent.
        \item Let the columns of \( A \) be denoted by \( \bma_1, \bma_2, \bma_3, \bma_4,
              \text{ and } \bma_5 \). \\ Determine whether each set is linearly independent:
              \begin{enumerate}[label=(\roman*)]
                  \item \( \{ \bma_1, \bma_2, \bma_4 \} \)
                  \item \( \{ \bma_1, \bma_2, \bma_3 \} \)
                  \item \( \{ \bma_1, \bma_3, \bma_5 \} \)
              \end{enumerate}
    \end{enumerate}
    ,where \[
        \begin{align*}
            A & = \begin{bmatrix}
                      - 2 & - 5 & 8    & 0 & - 17 \\
                      1   & 3   & - 5  & 1 & 5    \\
                      3   & 11  & - 19 & 7 & 1    \\
                      1   & 7   & - 13 & 5 & - 3
                  \end{bmatrix} \\
            B & = \begin{bmatrix}
                      1 & 0 & 1   & 0 & 1   \\
                      0 & 1 & - 2 & 0 & 3\, \\
                      0 & 0 & 0   & 1 & - 5 \\
                      0 & 0 & 0   & 0 & 0   \\
                  \end{bmatrix}
        \end{align*}
    \]}

\solve{
    \begin{enumerate}[label=(\alph*)]
        \item \( \mathrm{rank}(A) = \mathrm{rank}(B) = 3,\) \\
              \( \mathrm{nullity}(A) = \mathrm{nullity}(B) = 2 \). \\
        \item \( \mrN (A) = \mrN (B) = \mathrm{span}(S_N)\), where \[
                  S_N = \left\{
                  \begin{bmatrix}
                      -1 \\
                      2  \\
                      1  \\
                      0  \\
                      0  \\
                  \end{bmatrix},
                  \begin{bmatrix}
                      - 1 \\
                      - 3 \\
                      0   \\
                      5   \\
                      1   \\
                  \end{bmatrix}
                  \right\}.
              \]
        \item \( \mrR (A) = \mrR (B) = \mathrm{span}(S_R)\), where \[
                  S_R = \left\{
                  \begin{bmatrix}
                      1 & 0 & 1 & 0 & 1 \\
                  \end{bmatrix},
                  \begin{bmatrix}
                      0 & 1 & -2 & 0 & 3 \\
                  \end{bmatrix},
                  \begin{bmatrix}
                      0 & 0 & 0 & 1 & -5 \\
                  \end{bmatrix}
                  \right\}.
              \]
        \item \( \mrC (A) \) is spanned by the independent columns of \( A \), which can be found by the correspondant pivot columns of \( B \), \\
              \( \therefore \mrC (A) = \mathrm{span}(S_C)\), where \[
                  S_C = \left\{
                  \begin{bmatrix}
                      -2 \\
                      1  \\
                      3  \\
                      1  \\
                  \end{bmatrix},
                  \begin{bmatrix}
                      - 5 \\
                      3   \\
                      11  \\
                      7   \\
                  \end{bmatrix},
                  \begin{bmatrix}
                      0 \\
                      1 \\
                      7 \\
                      5
                  \end{bmatrix}
                  \right\}.
              \]
        \item The rows of \( A \) are not linearly independent, since the rank of \( A \) is
              less than the number of rows. \\
        \item We can easily check the answers by looking at the correspondant columns of \( B
              \)
              \begin{enumerate}[label=(\roman*)]
                  \item \( \{ \bma_1, \bma_2, \bma_4 \} \) is linearly independent. \\
                  \item \( \{ \bma_1, \bma_2, \bma_3 \} \) is not linearly independent. \\
                  \item \( \{ \bma_1, \bma_3, \bma_5 \} \) is not linearly independent. \\
              \end{enumerate}
    \end{enumerate}
}

\qs{}{(CH4.6 P206 Q62) Determine whether \[
        \bmb = \begin{bmatrix}
            - 9 \\ 11 \\ - 25
        \end{bmatrix}
    \] is in the column space of \[
        A = \begin{bmatrix}
            5   & 4 & 4   \\
            - 3 & 1 & - 2 \\
            1   & 0 & 8
        \end{bmatrix}.
    \] If it is, write \( \bmb \) as a linear combination of the column vectors of \(
    A \). }

\solve{
    \[ A \triangleq \begin{bmatrix}
            \bma_1 & \bma_2 & \bma_3
        \end{bmatrix}\]
    Do row opeartions on \( [\, A \,|\, \bmb \,] \):
    \[
        [\, A \,|\, \bmb \,] \sim \begin{bmatrix}
            1 & 0 & 0 & - 1 \\
            0 & 1 & 0 & 2\, \\
            0 & 0 & 1 & - 3
        \end{bmatrix}.
    \]
    \( \therefore \bmb = -\bma_1 + 2\bma_2 - 3\bma_3 \).
}

\end{document}