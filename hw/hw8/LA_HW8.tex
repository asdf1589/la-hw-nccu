\documentclass{report}

\usepackage{ctex}

\usepackage[tmargin=2cm,rmargin=1in,lmargin=1in,margin=0.85in,bmargin=2cm,footskip=.2in]{geometry}
\usepackage{amsmath,amsfonts,amsthm,amssymb,mathtools}
\usepackage[varbb]{newpxmath}
\usepackage{xfrac}
\usepackage[makeroom]{cancel}
\usepackage{mathtools}
\usepackage{bookmark}
\usepackage{enumitem}
\usepackage{hyperref,theoremref}
\hypersetup{
	pdftitle={assignment},
	colorlinks=true, linkcolor=doc!90,
	bookmarksnumbered=true,
	bookmarksopen=true
}
\usepackage[most,many,breakable]{tcolorbox}
\usepackage{xcolor}
\usepackage{varwidth}
\usepackage{varwidth}
\usepackage{etoolbox}
%\usepackage{authblk}
\usepackage{nameref}
\usepackage{multicol,array}
\usepackage[ruled,vlined,linesnumbered]{algorithm2e}
\usepackage{comment} % enables the use of multi-line comments (\ifx \fi) 
\usepackage{import}
\usepackage{xifthen}
\usepackage{pdfpages}
\usepackage{transparent}
\usepackage{chngcntr}
\usepackage{tikz}
\usepackage{titletoc}

\newcommand\mycommfont[1]{\footnotesize\ttfamily\textcolor{blue}{#1}}
\SetCommentSty{mycommfont}
\newcommand{\incfig}[1]{%
    \def\svgwidth{\columnwidth}
    \import{./figures/}{#1.pdf_tex}
}

\usepackage{tikzsymbols}
\tikzset{
	symbol/.style={
			draw=none,
			every to/.append style={
					edge node={node [sloped, allow upside down, auto=false]{$#1$}}}
		}
}
\tikzstyle{c} = [circle,fill=black,scale=0.5]
\tikzstyle{b} = [draw, thick, black, -]
\tikzset{
    vertex/.style={
        circle,
        draw,
        minimum size=6mm,
        inner sep=0pt
    }
}
\renewcommand\qedsymbol{$\Laughey$}

%\usepackage{import}
%\usepackage{xifthen}
%\usepackage{pdfpages}
%\usepackage{transparent}


%%%%%%%%%%%%%%%%%%%%%%%%%%%%%%
% SELF MADE COLORS
%%%%%%%%%%%%%%%%%%%%%%%%%%%%%%

\definecolor{doc}{RGB}{0,60,110}
\definecolor{myg}{RGB}{56, 140, 70}
\definecolor{myb}{RGB}{45, 111, 177}
\definecolor{myr}{RGB}{199, 68, 64}
\definecolor{mytheorembg}{HTML}{F2F2F9}
\definecolor{mytheoremfr}{HTML}{00007B}
\definecolor{mylemmabg}{HTML}{FFFAF8}
\definecolor{mylemmafr}{HTML}{983b0f}
\definecolor{mypropbg}{HTML}{f2fbfc}
\definecolor{mypropfr}{HTML}{191971}
\definecolor{myexamplebg}{HTML}{F2FBF8}
\definecolor{myexamplefr}{HTML}{88D6D1}
\definecolor{myexampleti}{HTML}{2A7F7F}
\definecolor{mydefinitbg}{HTML}{E5E5FF}
\definecolor{mydefinitfr}{HTML}{3F3FA3}
\definecolor{notesgreen}{RGB}{0,162,0}
\definecolor{myp}{RGB}{197, 92, 212}
\definecolor{mygr}{HTML}{2C3338}
\definecolor{myred}{RGB}{127,0,0}
\definecolor{myyellow}{RGB}{169,121,69}
\definecolor{myexercisebg}{HTML}{F2FBF8}
\definecolor{myexercisefg}{HTML}{88D6D1}

%%%%%%%%%%%%%%%%%%%%%%%%%%%%
% TCOLORBOX SETUPS
%%%%%%%%%%%%%%%%%%%%%%%%%%%%

\setlength{\parindent}{1cm}
%================================
% THEOREM BOX
%================================

\tcbuselibrary{theorems,skins,hooks}
\newtcbtheorem[number within=section]{Theorem}{Theorem}
{%
	enhanced,
	breakable,
	colback = mytheorembg,
	frame hidden,
	boxrule = 0sp,
	borderline west = {2pt}{0pt}{mytheoremfr},
	sharp corners,
	detach title,
	before upper = \tcbtitle\par\smallskip,
	coltitle = mytheoremfr,
	fonttitle = \bfseries\sffamily,
	description font = \mdseries,
	separator sign none,
	segmentation style={solid, mytheoremfr},
}
{th}

\tcbuselibrary{theorems,skins,hooks}
\newtcbtheorem[number within=chapter]{theorem}{Theorem}
{%
	enhanced,
	breakable,
	colback = mytheorembg,
	frame hidden,
	boxrule = 0sp,
	borderline west = {2pt}{0pt}{mytheoremfr},
	sharp corners,
	detach title,
	before upper = \tcbtitle\par\smallskip,
	coltitle = mytheoremfr,
	fonttitle = \bfseries\sffamily,
	description font = \mdseries,
	separator sign none,
	segmentation style={solid, mytheoremfr},
}
{th}


\tcbuselibrary{theorems,skins,hooks}
\newtcolorbox{Theoremcon}
{%
	enhanced
	,breakable
	,colback = mytheorembg
	,frame hidden
	,boxrule = 0sp
	,borderline west = {2pt}{0pt}{mytheoremfr}
	,sharp corners
	,description font = \mdseries
	,separator sign none
}

%================================
% Corollery
%================================
\tcbuselibrary{theorems,skins,hooks}
\newtcbtheorem[number within=section]{Corollary}{Corollary}
{%
	enhanced
	,breakable
	,colback = myp!10
	,frame hidden
	,boxrule = 0sp
	,borderline west = {2pt}{0pt}{myp!85!black}
	,sharp corners
	,detach title
	,before upper = \tcbtitle\par\smallskip
	,coltitle = myp!85!black
	,fonttitle = \bfseries\sffamily
	,description font = \mdseries
	,separator sign none
	,segmentation style={solid, myp!85!black}
}
{th}
\tcbuselibrary{theorems,skins,hooks}
\newtcbtheorem[number within=chapter]{corollary}{Corollary}
{%
	enhanced
	,breakable
	,colback = myp!10
	,frame hidden
	,boxrule = 0sp
	,borderline west = {2pt}{0pt}{myp!85!black}
	,sharp corners
	,detach title
	,before upper = \tcbtitle\par\smallskip
	,coltitle = myp!85!black
	,fonttitle = \bfseries\sffamily
	,description font = \mdseries
	,separator sign none
	,segmentation style={solid, myp!85!black}
}
{th}


%================================
% LEMMA
%================================

\tcbuselibrary{theorems,skins,hooks}
\newtcbtheorem[number within=section]{Lemma}{Lemma}
{%
	enhanced,
	breakable,
	colback = mylemmabg,
	frame hidden,
	boxrule = 0sp,
	borderline west = {2pt}{0pt}{mylemmafr},
	sharp corners,
	detach title,
	before upper = \tcbtitle\par\smallskip,
	coltitle = mylemmafr,
	fonttitle = \bfseries\sffamily,
	description font = \mdseries,
	separator sign none,
	segmentation style={solid, mylemmafr},
}
{th}

\tcbuselibrary{theorems,skins,hooks}
\newtcbtheorem[number within=chapter]{lemma}{lemma}
{%
	enhanced,
	breakable,
	colback = mylemmabg,
	frame hidden,
	boxrule = 0sp,
	borderline west = {2pt}{0pt}{mylemmafr},
	sharp corners,
	detach title,
	before upper = \tcbtitle\par\smallskip,
	coltitle = mylemmafr,
	fonttitle = \bfseries\sffamily,
	description font = \mdseries,
	separator sign none,
	segmentation style={solid, mylemmafr},
}
{th}

%================================
% Exercise
%================================

\tcbuselibrary{theorems,skins,hooks}
\newtcbtheorem[number within=section]{Exercise}{Exercise}
{%
	enhanced,
	breakable,
	colback = myexercisebg,
	frame hidden,
	boxrule = 0sp,
	borderline west = {2pt}{0pt}{myexercisefg},
	sharp corners,
	detach title,
	before upper = \tcbtitle\par\smallskip,
	coltitle = myexercisefg,
	fonttitle = \bfseries\sffamily,
	description font = \mdseries,
	separator sign none,
	segmentation style={solid, myexercisefg},
}
{th}

\tcbuselibrary{theorems,skins,hooks}
\newtcbtheorem[number within=chapter]{exercise}{Exercise}
{%
	enhanced,
	breakable,
	colback = myexercisebg,
	frame hidden,
	boxrule = 0sp,
	borderline west = {2pt}{0pt}{myexercisefg},
	sharp corners,
	detach title,
	before upper = \tcbtitle\par\smallskip,
	coltitle = myexercisefg,
	fonttitle = \bfseries\sffamily,
	description font = \mdseries,
	separator sign none,
	segmentation style={solid, myexercisefg},
}
{th}


%================================
% PROPOSITION
%================================

\tcbuselibrary{theorems,skins,hooks}
\newtcbtheorem[number within=section]{Prop}{Proposition}
{%
	enhanced,
	breakable,
	colback = mypropbg,
	frame hidden,
	boxrule = 0sp,
	borderline west = {2pt}{0pt}{mypropfr},
	sharp corners,
	detach title,
	before upper = \tcbtitle\par\smallskip,
	coltitle = mypropfr,
	fonttitle = \bfseries\sffamily,
	description font = \mdseries,
	separator sign none,
	segmentation style={solid, mypropfr},
}
{th}

\tcbuselibrary{theorems,skins,hooks}
\newtcbtheorem[number within=chapter]{prop}{Proposition}
{%
	enhanced,
	breakable,
	colback = mypropbg,
	frame hidden,
	boxrule = 0sp,
	borderline west = {2pt}{0pt}{mypropfr},
	sharp corners,
	detach title,
	before upper = \tcbtitle\par\smallskip,
	coltitle = mypropfr,
	fonttitle = \bfseries\sffamily,
	description font = \mdseries,
	separator sign none,
	segmentation style={solid, mypropfr},
}
{th}


%================================
% CLAIM
%================================

\tcbuselibrary{theorems,skins,hooks}
\newtcbtheorem[number within=section]{claim}{Claim}
{%
	enhanced
	,breakable
	,colback = myg!10
	,frame hidden
	,boxrule = 0sp
	,borderline west = {2pt}{0pt}{myg}
	,sharp corners
	,detach title
	,before upper = \tcbtitle\par\smallskip
	,coltitle = myg!85!black
	,fonttitle = \bfseries\sffamily
	,description font = \mdseries
	,separator sign none
	,segmentation style={solid, myg!85!black}
}
{th}



%================================
% EXAMPLE BOX
%================================

\newtcbtheorem[number within=section]{Example}{Example}
{%
	colback = myexamplebg
	,breakable
	,colframe = myexamplefr
	,coltitle = myexampleti
	,boxrule = 1pt
	,sharp corners
	,detach title
	,before upper=\tcbtitle\par\smallskip
	,fonttitle = \bfseries
	,description font = \mdseries
	,separator sign none
	,description delimiters parenthesis
}
{ex}

\newtcbtheorem[number within=chapter]{example}{Example}
{%
	colback = myexamplebg
	,breakable
	,colframe = myexamplefr
	,coltitle = myexampleti
	,boxrule = 1pt
	,sharp corners
	,detach title
	,before upper=\tcbtitle\par\smallskip
	,fonttitle = \bfseries
	,description font = \mdseries
	,separator sign none
	,description delimiters parenthesis
}
{ex}

%================================
% DEFINITION BOX
%================================

\newtcbtheorem[number within=section]{Definition}{Definition}{enhanced,
	before skip=2mm,after skip=2mm, colback=red!5,colframe=red!80!black,boxrule=0.5mm,
	attach boxed title to top left={xshift=1cm,yshift*=1mm-\tcboxedtitleheight}, varwidth boxed title*=-3cm,
	boxed title style={frame code={
					\path[fill=tcbcolback]
					([yshift=-1mm,xshift=-1mm]frame.north west)
					arc[start angle=0,end angle=180,radius=1mm]
					([yshift=-1mm,xshift=1mm]frame.north east)
					arc[start angle=180,end angle=0,radius=1mm];
					\path[left color=tcbcolback!60!black,right color=tcbcolback!60!black,
						middle color=tcbcolback!80!black]
					([xshift=-2mm]frame.north west) -- ([xshift=2mm]frame.north east)
					[rounded corners=1mm]-- ([xshift=1mm,yshift=-1mm]frame.north east)
					-- (frame.south east) -- (frame.south west)
					-- ([xshift=-1mm,yshift=-1mm]frame.north west)
					[sharp corners]-- cycle;
				},interior engine=empty,
		},
	fonttitle=\bfseries,
	title={#2},#1}{def}
\newtcbtheorem[number within=chapter]{definition}{Definition}{enhanced,
	before skip=2mm,after skip=2mm, colback=red!5,colframe=red!80!black,boxrule=0.5mm,
	attach boxed title to top left={xshift=1cm,yshift*=1mm-\tcboxedtitleheight}, varwidth boxed title*=-3cm,
	boxed title style={frame code={
					\path[fill=tcbcolback]
					([yshift=-1mm,xshift=-1mm]frame.north west)
					arc[start angle=0,end angle=180,radius=1mm]
					([yshift=-1mm,xshift=1mm]frame.north east)
					arc[start angle=180,end angle=0,radius=1mm];
					\path[left color=tcbcolback!60!black,right color=tcbcolback!60!black,
						middle color=tcbcolback!80!black]
					([xshift=-2mm]frame.north west) -- ([xshift=2mm]frame.north east)
					[rounded corners=1mm]-- ([xshift=1mm,yshift=-1mm]frame.north east)
					-- (frame.south east) -- (frame.south west)
					-- ([xshift=-1mm,yshift=-1mm]frame.north west)
					[sharp corners]-- cycle;
				},interior engine=empty,
		},
	fonttitle=\bfseries,
	title={#2},#1}{def}


%================================
% EXERCISE BOX
%================================

\newcounter{questioncounter}
\counterwithin{questioncounter}{chapter}
% \counterwithin{questioncounter}{section}

\makeatletter
\newtcbtheorem[use counter=questioncounter]{question}{Question}{enhanced,
	breakable,
	colback=white,
	colframe=myb!80!black,
	attach boxed title to top left={yshift*=-\tcboxedtitleheight},
	fonttitle=\bfseries,
	title={#2},
	boxed title size=title,
	boxed title style={%
			sharp corners,
			rounded corners=northwest,
			colback=tcbcolframe,
			boxrule=0pt,
		},
	underlay boxed title={%
			\path[fill=tcbcolframe] (title.south west)--(title.south east)
			to[out=0, in=180] ([xshift=5mm]title.east)--
			(title.center-|frame.east)
			[rounded corners=\kvtcb@arc] |-
			(frame.north) -| cycle;
		},
	#1
}{def}
\makeatother

%================================
% SOLUTION BOX
%================================

\makeatletter
\newtcolorbox{solution}{enhanced,
	breakable,
	colback=white,
	colframe=myg!80!black,
	attach boxed title to top left={yshift*=-\tcboxedtitleheight},
	title=Solution,
	boxed title size=title,
	boxed title style={%
			sharp corners,
			rounded corners=northwest,
			colback=tcbcolframe,
			boxrule=0pt,
		},
	underlay boxed title={%
			\path[fill=tcbcolframe] (title.south west)--(title.south east)
			to[out=0, in=180] ([xshift=5mm]title.east)--
			(title.center-|frame.east)
			[rounded corners=\kvtcb@arc] |-
			(frame.north) -| cycle;
		},
}
\makeatother

%================================
% Question BOX
%================================

\makeatletter
\newtcbtheorem{qstion}{Question}{enhanced,
    breakable,
    colback=white,
    colframe=mygr,
    attach boxed title to top left={yshift*=-\tcboxedtitleheight},
    fonttitle=\bfseries,
    title={#2},
    boxed title size=title,
    boxed title style={%
            sharp corners,
            rounded corners=northwest,
            colback=tcbcolframe,
            boxrule=0pt
        },
    underlay boxed title={%
            \path[fill=tcbcolframe](title.south west)--(title.south east)
            to[out=0,in=180]([xshift=5mm]title.east)--
            (title.center-|frame.east)
            [rounded corners=\kvtcb@arc]|-
            (frame.north)-|cycle;
        },
    #1
}{def}
\makeatother

\newtcbtheorem[number within=chapter]{wconc}{Wrong Concept}{
	breakable,
	enhanced,
	colback=white,
	colframe=myr,
	arc=0pt,
	outer arc=0pt,
	fonttitle=\bfseries\sffamily\large,
	colbacktitle=myr,
	attach boxed title to top left={},
	boxed title style={
			enhanced,
			skin=enhancedfirst jigsaw,
			arc=3pt,
			bottom=0pt,
			interior style={fill=myr}
		},
	#1
}{def}


%================================
% NOTE BOX
%================================

\usetikzlibrary{arrows,calc,shadows.blur}
\tcbuselibrary{skins}
\newtcolorbox{note}[1][]{%
	enhanced jigsaw,
	colback=gray!20!white,%
	colframe=gray!80!black,
	size=small,
	boxrule=1pt,
	title=\textbf{Note:-},
	halign title=flush center,
	coltitle=black,
	breakable,
	drop shadow=black!50!white,
	attach boxed title to top left={xshift=1cm,yshift=-\tcboxedtitleheight/2,yshifttext=-\tcboxedtitleheight/2},
	minipage boxed title=1.5cm,
	boxed title style={%
			colback=white,
			size=fbox,
			boxrule=1pt,
			boxsep=2pt,
			underlay={%
					\coordinate (dotA) at ($(interior.west) + (-0.5pt,0)$);
					\coordinate (dotB) at ($(interior.east) + (0.5pt,0)$);
					\begin{scope}
						\clip (interior.north west) rectangle ([xshift=3ex]interior.east);
						\filldraw [white, blur shadow={shadow opacity=60, shadow yshift=-.75ex}, rounded corners=2pt] (interior.north west) rectangle (interior.south east);
					\end{scope}
					\begin{scope}[gray!80!black]
						\fill (dotA) circle (2pt);
						\fill (dotB) circle (2pt);
					\end{scope}
				},
		},
	#1,
}

%%%%%%%%%%%%%%%%%%%%%%%%%%%%%%
% SELF MADE COMMANDS
%%%%%%%%%%%%%%%%%%%%%%%%%%%%%%

\newcommand{\thm}[2]{\begin{Theorem}{#1}{}#2\end{Theorem}}
\newcommand{\cor}[2]{\begin{Corollary}{#1}{}#2\end{Corollary}}
\newcommand{\mlemma}[2]{\begin{Lemma}{#1}{}#2\end{Lemma}}
\newcommand{\mer}[2]{\begin{Exercise}{#1}{}#2\end{Exercise}}
\newcommand{\mprop}[2]{\begin{Prop}{#1}{}#2\end{Prop}}
\newcommand{\clm}[3]{\begin{claim}{#1}{#2}#3\end{claim}}
\newcommand{\wc}[2]{\begin{wconc}{#1}{}\setlength{\parindent}{1cm}#2\end{wconc}}
\newcommand{\thmcon}[1]{\begin{Theoremcon}{#1}\end{Theoremcon}}
\newcommand{\ex}[2]{\begin{Example}{#1}{}#2\end{Example}}
\newcommand{\dfn}[2]{\begin{Definition}[colbacktitle=red!75!black]{#1}{}#2\end{Definition}}
\newcommand{\dfnc}[2]{\begin{definition}[colbacktitle=red!75!black]{#1}{}#2\end{definition}}
\newcommand{\qs}[2]{\begin{question}{#1}{}#2\end{question}}
\newcommand{\pf}[2]{\begin{myproof}[#1]#2\end{myproof}}
\newcommand{\nt}[1]{\begin{note}#1\end{note}}

\newcommand*\circled[1]{\tikz[baseline=(char.base)]{
		\node[shape=circle,draw,inner sep=1pt] (char) {#1};}}
\newcommand\getcurrentref[1]{%
	\ifnumequal{\value{#1}}{0}
	{??}
	{\the\value{#1}}%
}
\newcommand{\getCurrentSectionNumber}{\getcurrentref{section}}
\newenvironment{myproof}[1][\proofname]{%
	\proof[\bfseries #1: ]%
}{\endproof}

\newcommand{\mclm}[2]{\begin{myclaim}[#1]#2\end{myclaim}}
\newenvironment{myclaim}[1][\claimname]{\proof[\bfseries #1: ]}{}
\newenvironment{iclaim}[1][\claimname]{\bfseries #1\mdseries:}{}
\newcommand{\iclm}[2]{\begin{iclaim}[#1]#2\end{iclaim}}

\newcounter{mylabelcounter}

\makeatletter
\newcommand{\setword}[2]{%
	\phantomsection
	#1\def\@currentlabel{\unexpanded{#1}}\label{#2}%
}
\makeatother

% deliminators
\DeclarePairedDelimiter{\abs}{\lvert}{\rvert}
\DeclarePairedDelimiter{\norm}{\lVert}{\rVert}

\DeclarePairedDelimiter{\ceil}{\lceil}{\rceil}
\DeclarePairedDelimiter{\floor}{\lfloor}{\rfloor}
\DeclarePairedDelimiter{\round}{\lfloor}{\rceil}

\newsavebox\diffdbox
\newcommand{\slantedromand}{{\mathpalette\makesl{d}}}
\newcommand{\makesl}[2]{%
\begingroup
\sbox{\diffdbox}{$\mathsurround=0pt#1\mathrm{#2}$}%
\pdfsave
\pdfsetmatrix{1 0 0.2 1}%
\rlap{\usebox{\diffdbox}}%
\pdfrestore
\hskip\wd\diffdbox
\endgroup
}
\newcommand{\dd}[1][]{\ensuremath{\mathop{}\!\ifstrempty{#1}{%
\slantedromand\@ifnextchar^{\hspace{0.2ex}}{\hspace{0.1ex}}}%
{\slantedromand\hspace{0.2ex}^{#1}}}}
\ProvideDocumentCommand\dv{o m g}{%
  \ensuremath{%
    \IfValueTF{#3}{%
      \IfNoValueTF{#1}{%
        \frac{\dd #2}{\dd #3}%
      }{%
        \frac{\dd^{#1} #2}{\dd #3^{#1}}%
      }%
    }{%
      \IfNoValueTF{#1}{%
        \frac{\dd}{\dd #2}%
      }{%
        \frac{\dd^{#1}}{\dd #2^{#1}}%
      }%
    }%
  }%
}
\providecommand*{\pdv}[3][]{\frac{\partial^{#1}#2}{\partial#3^{#1}}}
%  - others
\DeclareMathOperator{\Lap}{\mathcal{L}}
\DeclareMathOperator{\Var}{Var} % varience
\DeclareMathOperator{\Cov}{Cov} % covarience
\DeclareMathOperator{\E}{E} % expected

% Since the amsthm package isn't loaded

% I prefer the slanted \leq
\let\oldleq\leq % save them in case they're every wanted
\let\oldgeq\geq
\renewcommand{\leq}{\leqslant}
\renewcommand{\geq}{\geqslant}

%%%%%%%%%%%%%%%%%%%%%%%%%%%%%%%%%%%%%%%%%%%
% TABLE OF CONTENTS
%%%%%%%%%%%%%%%%%%%%%%%%%%%%%%%%%%%%%%%%%%%

\contentsmargin{0cm}
\titlecontents{chapter}[3.7pc]
{\addvspace{30pt}%
	\begin{tikzpicture}[remember picture, overlay]%
		\draw[fill=doc!60,draw=doc!60] (-7,-.1) rectangle (-0.9,.5);%
		\pgftext[left,x=-3.7cm,y=0.2cm]{\color{white}\Large\sc\bfseries Chapter\ \thecontentslabel};%
	\end{tikzpicture}\color{doc!60}\large\sc\bfseries}%
{}
{}
{\;\titlerule\;\large\sc\bfseries Page \thecontentspage
	\begin{tikzpicture}[remember picture, overlay]
		\draw[fill=doc!60,draw=doc!60] (2pt,0) rectangle (4,0.1pt);
	\end{tikzpicture}}%
\titlecontents{section}[3.7pc]
{\addvspace{2pt}}
{\contentslabel[\thecontentslabel]{2pc}}
{}
{\hfill\small \thecontentspage}
[]
\titlecontents*{subsection}[3.7pc]
{\addvspace{-1pt}\small}
{}
{}
{\ --- \small\thecontentspage}
[ \textbullet\ ][]

\makeatletter
\renewcommand{\tableofcontents}{%
	\chapter*{%
	  \vspace*{-20\p@}%
	  \begin{tikzpicture}[remember picture, overlay]%
		  \pgftext[right,x=15cm,y=0.2cm]{\color{doc!60}\Huge\sc\bfseries \contentsname};%
		  \draw[fill=doc!60,draw=doc!60] (13,-.75) rectangle (20,1);%
		  \clip (13,-.75) rectangle (20,1);
		  \pgftext[right,x=15cm,y=0.2cm]{\color{white}\Huge\sc\bfseries \contentsname};%
	  \end{tikzpicture}}%
	\@starttoc{toc}}
\makeatother

\newcommand{\eps}{\epsilon}
\newcommand{\veps}{\varepsilon}
\newcommand{\Qed}{\begin{flushright}\qed\end{flushright}}

\newcommand{\parinn}{\setlength{\parindent}{1cm}}
\newcommand{\parinf}{\setlength{\parindent}{0cm}}

% \newcommand{\norm}{\|\cdot\|}
\newcommand{\inorm}{\norm_{\infty}}
\newcommand{\opensets}{\{V_{\alpha}\}_{\alpha\in I}}
\newcommand{\oset}{V_{\alpha}}
\newcommand{\opset}[1]{V_{\alpha_{#1}}}
\newcommand{\lub}{\text{lub}}
\newcommand{\del}[2]{\frac{\partial #1}{\partial #2}}
\newcommand{\Del}[3]{\frac{\partial^{#1} #2}{\partial^{#1} #3}}
\newcommand{\deld}[2]{\dfrac{\partial #1}{\partial #2}}
\newcommand{\Deld}[3]{\dfrac{\partial^{#1} #2}{\partial^{#1} #3}}
\newcommand{\der}[2]{\frac{\mathrm{d} #1}{\mathrm{d} #2}}
% \newcommand{\ddd}[3]{\frac{\mathrm{d}^{#3} #1}{\mathrm{d}^{#3} #2}}
\newcommand{\lm}{\lambda}
\newcommand{\uin}{\mathbin{\rotatebox[origin=c]{90}{$\in$}}}
\newcommand{\usubset}{\mathbin{\rotatebox[origin=c]{90}{$\subset$}}}
\newcommand{\lt}{\left}
\newcommand{\rt}{\right}
\newcommand{\bs}[1]{\boldsymbol{#1}}
\newcommand{\exs}{\exists}
\newcommand{\st}{\strut}
\newcommand{\dps}[1]{\displaystyle{#1}}
\newcommand{\id}{\text{id}}


\newcommand{\sol}{\setlength{\parindent}{0cm}\textbf{\textit{Solution:}}\setlength{\parindent}{1cm} }
\newcommand{\solve}[1]{\setlength{\parindent}{0cm}\textbf{\textit{Solution: }}\setlength{\parindent}{1cm}#1 \Qed}

% number sets
\newcommand{\RR}[1][]{\ensuremath{\ifstrempty{#1}{\mathbb{R}}{\mathbb{R}^{#1}}}}
\newcommand{\NN}[1][]{\ensuremath{\ifstrempty{#1}{\mathbb{N}}{\mathbb{N}^{#1}}}}
\newcommand{\ZZ}[1][]{\ensuremath{\ifstrempty{#1}{\mathbb{Z}}{\mathbb{Z}^{#1}}}}
\newcommand{\QQ}[1][]{\ensuremath{\ifstrempty{#1}{\mathbb{Q}}{\mathbb{Q}^{#1}}}}
\newcommand{\CC}[1][]{\ensuremath{\ifstrempty{#1}{\mathbb{C}}{\mathbb{C}^{#1}}}}
\newcommand{\PP}[1][]{\ensuremath{\ifstrempty{#1}{\mathbb{P}}{\mathbb{P}^{#1}}}}
\newcommand{\HH}[1][]{\ensuremath{\ifstrempty{#1}{\mathbb{H}}{\mathbb{H}^{#1}}}}
\newcommand{\FF}[1][]{\ensuremath{\ifstrempty{#1}{\mathbb{F}}{\mathbb{F}^{#1}}}}
% expected value
\newcommand{\EE}{\ensuremath{\mathbb{E}}}

%---------------------------------------
% BlackBoard Math Fonts :-
%---------------------------------------

%Captital Letters
\newcommand{\bbA}{\mathbb{A}}	\newcommand{\bbB}{\mathbb{B}}
\newcommand{\bbC}{\mathbb{C}}	\newcommand{\bbD}{\mathbb{D}}
\newcommand{\bbE}{\mathbb{E}}	\newcommand{\bbF}{\mathbb{F}}
\newcommand{\bbG}{\mathbb{G}}	\newcommand{\bbH}{\mathbb{H}}
\newcommand{\bbI}{\mathbb{I}}	\newcommand{\bbJ}{\mathbb{J}}
\newcommand{\bbK}{\mathbb{K}}	\newcommand{\bbL}{\mathbb{L}}
\newcommand{\bbM}{\mathbb{M}}	\newcommand{\bbN}{\mathbb{N}}
\newcommand{\bbO}{\mathbb{O}}	\newcommand{\bbP}{\mathbb{P}}
\newcommand{\bbQ}{\mathbb{Q}}	\newcommand{\bbR}{\mathbb{R}}
\newcommand{\bbS}{\mathbb{S}}	\newcommand{\bbT}{\mathbb{T}}
\newcommand{\bbU}{\mathbb{U}}	\newcommand{\bbV}{\mathbb{V}}
\newcommand{\bbW}{\mathbb{W}}	\newcommand{\bbX}{\mathbb{X}}
\newcommand{\bbY}{\mathbb{Y}}	\newcommand{\bbZ}{\mathbb{Z}}

%---------------------------------------
% MathCal Fonts :-
%---------------------------------------

%Captital Letters
\newcommand{\mcA}{\mathcal{A}}	\newcommand{\mcB}{\mathcal{B}}
\newcommand{\mcC}{\mathcal{C}}	\newcommand{\mcD}{\mathcal{D}}
\newcommand{\mcE}{\mathcal{E}}	\newcommand{\mcF}{\mathcal{F}}
\newcommand{\mcG}{\mathcal{G}}	\newcommand{\mcH}{\mathcal{H}}
\newcommand{\mcI}{\mathcal{I}}	\newcommand{\mcJ}{\mathcal{J}}
\newcommand{\mcK}{\mathcal{K}}	\newcommand{\mcL}{\mathcal{L}}
\newcommand{\mcM}{\mathcal{M}}	\newcommand{\mcN}{\mathcal{N}}
\newcommand{\mcO}{\mathcal{O}}	\newcommand{\mcP}{\mathcal{P}}
\newcommand{\mcQ}{\mathcal{Q}}	\newcommand{\mcR}{\mathcal{R}}
\newcommand{\mcS}{\mathcal{S}}	\newcommand{\mcT}{\mathcal{T}}
\newcommand{\mcU}{\mathcal{U}}	\newcommand{\mcV}{\mathcal{V}}
\newcommand{\mcW}{\mathcal{W}}	\newcommand{\mcX}{\mathcal{X}}
\newcommand{\mcY}{\mathcal{Y}}	\newcommand{\mcZ}{\mathcal{Z}}



%---------------------------------------
% Bold Math Fonts :-
%---------------------------------------

%Captital Letters
\newcommand{\bmA}{\boldsymbol{A}}	\newcommand{\bmB}{\boldsymbol{B}}
\newcommand{\bmC}{\boldsymbol{C}}	\newcommand{\bmD}{\boldsymbol{D}}
\newcommand{\bmE}{\boldsymbol{E}}	\newcommand{\bmF}{\boldsymbol{F}}
\newcommand{\bmG}{\boldsymbol{G}}	\newcommand{\bmH}{\boldsymbol{H}}
\newcommand{\bmI}{\boldsymbol{I}}	\newcommand{\bmJ}{\boldsymbol{J}}
\newcommand{\bmK}{\boldsymbol{K}}	\newcommand{\bmL}{\boldsymbol{L}}
\newcommand{\bmM}{\boldsymbol{M}}	\newcommand{\bmN}{\boldsymbol{N}}
\newcommand{\bmO}{\boldsymbol{O}}	\newcommand{\bmP}{\boldsymbol{P}}
\newcommand{\bmQ}{\boldsymbol{Q}}	\newcommand{\bmR}{\boldsymbol{R}}
\newcommand{\bmS}{\boldsymbol{S}}	\newcommand{\bmT}{\boldsymbol{T}}
\newcommand{\bmU}{\boldsymbol{U}}	\newcommand{\bmV}{\boldsymbol{V}}
\newcommand{\bmW}{\boldsymbol{W}}	\newcommand{\bmX}{\boldsymbol{X}}
\newcommand{\bmY}{\boldsymbol{Y}}	\newcommand{\bmZ}{\boldsymbol{Z}}
%Small Letters
\newcommand{\bma}{\boldsymbol{a}}	\newcommand{\bmb}{\boldsymbol{b}}
\newcommand{\bmc}{\boldsymbol{c}}	\newcommand{\bmd}{\boldsymbol{d}}
\newcommand{\bme}{\boldsymbol{e}}	\newcommand{\bmf}{\boldsymbol{f}}
\newcommand{\bmg}{\boldsymbol{g}}	\newcommand{\bmh}{\boldsymbol{h}}
\newcommand{\bmi}{\boldsymbol{i}}	\newcommand{\bmj}{\boldsymbol{j}}
\newcommand{\bmk}{\boldsymbol{k}}	\newcommand{\bml}{\boldsymbol{l}}
\newcommand{\bmm}{\boldsymbol{m}}	\newcommand{\bmn}{\boldsymbol{n}}
\newcommand{\bmo}{\boldsymbol{o}}	\newcommand{\bmp}{\boldsymbol{p}}
\newcommand{\bmq}{\boldsymbol{q}}	\newcommand{\bmr}{\boldsymbol{r}}
\newcommand{\bms}{\boldsymbol{s}}	\newcommand{\bmt}{\boldsymbol{t}}
\newcommand{\bmu}{\boldsymbol{u}}	\newcommand{\bmv}{\boldsymbol{v}}
\newcommand{\bmw}{\boldsymbol{w}}	\newcommand{\bmx}{\boldsymbol{x}}
\newcommand{\bmy}{\boldsymbol{y}}	\newcommand{\bmz}{\boldsymbol{z}}

%---------------------------------------
% Scr Math Fonts :-
%---------------------------------------

\newcommand{\sA}{{\mathscr{A}}}   \newcommand{\sB}{{\mathscr{B}}}
\newcommand{\sC}{{\mathscr{C}}}   \newcommand{\sD}{{\mathscr{D}}}
\newcommand{\sE}{{\mathscr{E}}}   \newcommand{\sF}{{\mathscr{F}}}
\newcommand{\sG}{{\mathscr{G}}}   \newcommand{\sH}{{\mathscr{H}}}
\newcommand{\sI}{{\mathscr{I}}}   \newcommand{\sJ}{{\mathscr{J}}}
\newcommand{\sK}{{\mathscr{K}}}   \newcommand{\sL}{{\mathscr{L}}}
\newcommand{\sM}{{\mathscr{M}}}   \newcommand{\sN}{{\mathscr{N}}}
\newcommand{\sO}{{\mathscr{O}}}   \newcommand{\sP}{{\mathscr{P}}}
\newcommand{\sQ}{{\mathscr{Q}}}   \newcommand{\sR}{{\mathscr{R}}}
\newcommand{\sS}{{\mathscr{S}}}   \newcommand{\sT}{{\mathscr{T}}}
\newcommand{\sU}{{\mathscr{U}}}   \newcommand{\sV}{{\mathscr{V}}}
\newcommand{\sW}{{\mathscr{W}}}   \newcommand{\sX}{{\mathscr{X}}}
\newcommand{\sY}{{\mathscr{Y}}}   \newcommand{\sZ}{{\mathscr{Z}}}


%---------------------------------------
% Math Fraktur Font
%---------------------------------------

%Captital Letters
\newcommand{\mfA}{\mathfrak{A}}	\newcommand{\mfB}{\mathfrak{B}}
\newcommand{\mfC}{\mathfrak{C}}	\newcommand{\mfD}{\mathfrak{D}}
\newcommand{\mfE}{\mathfrak{E}}	\newcommand{\mfF}{\mathfrak{F}}
\newcommand{\mfG}{\mathfrak{G}}	\newcommand{\mfH}{\mathfrak{H}}
\newcommand{\mfI}{\mathfrak{I}}	\newcommand{\mfJ}{\mathfrak{J}}
\newcommand{\mfK}{\mathfrak{K}}	\newcommand{\mfL}{\mathfrak{L}}
\newcommand{\mfM}{\mathfrak{M}}	\newcommand{\mfN}{\mathfrak{N}}
\newcommand{\mfO}{\mathfrak{O}}	\newcommand{\mfP}{\mathfrak{P}}
\newcommand{\mfQ}{\mathfrak{Q}}	\newcommand{\mfR}{\mathfrak{R}}
\newcommand{\mfS}{\mathfrak{S}}	\newcommand{\mfT}{\mathfrak{T}}
\newcommand{\mfU}{\mathfrak{U}}	\newcommand{\mfV}{\mathfrak{V}}
\newcommand{\mfW}{\mathfrak{W}}	\newcommand{\mfX}{\mathfrak{X}}
\newcommand{\mfY}{\mathfrak{Y}}	\newcommand{\mfZ}{\mathfrak{Z}}
%Small Letters
\newcommand{\mfa}{\mathfrak{a}}	\newcommand{\mfb}{\mathfrak{b}}
\newcommand{\mfc}{\mathfrak{c}}	\newcommand{\mfd}{\mathfrak{d}}
\newcommand{\mfe}{\mathfrak{e}}	\newcommand{\mff}{\mathfrak{f}}
\newcommand{\mfg}{\mathfrak{g}}	\newcommand{\mfh}{\mathfrak{h}}
\newcommand{\mfi}{\mathfrak{i}}	\newcommand{\mfj}{\mathfrak{j}}
\newcommand{\mfk}{\mathfrak{k}}	\newcommand{\mfl}{\mathfrak{l}}
\newcommand{\mfm}{\mathfrak{m}}	\newcommand{\mfn}{\mathfrak{n}}
\newcommand{\mfo}{\mathfrak{o}}	\newcommand{\mfp}{\mathfrak{p}}
\newcommand{\mfq}{\mathfrak{q}}	\newcommand{\mfr}{\mathfrak{r}}
\newcommand{\mfs}{\mathfrak{s}}	\newcommand{\mft}{\mathfrak{t}}
\newcommand{\mfu}{\mathfrak{u}}	\newcommand{\mfv}{\mathfrak{v}}
\newcommand{\mfw}{\mathfrak{w}}	\newcommand{\mfx}{\mathfrak{x}}
\newcommand{\mfy}{\mathfrak{y}}	\newcommand{\mfz}{\mathfrak{z}}


%---------------------------------------
% Math Roman Font
%---------------------------------------

%Captital Letters
\newcommand{\mrA}{\mathrm{A}}	\newcommand{\mrB}{\mathrm{B}}
\newcommand{\mrC}{\mathrm{C}}	\newcommand{\mrD}{\mathrm{D}}
\newcommand{\mrE}{\mathrm{E}}	\newcommand{\mrF}{\mathrm{F}}
\newcommand{\mrG}{\mathrm{G}}	\newcommand{\mrH}{\mathrm{H}}
\newcommand{\mrI}{\mathrm{I}}	\newcommand{\mrJ}{\mathrm{J}}
\newcommand{\mrK}{\mathrm{K}}	\newcommand{\mrL}{\mathrm{L}}
\newcommand{\mrM}{\mathrm{M}}	\newcommand{\mrN}{\mathrm{N}}
\newcommand{\mrO}{\mathrm{O}}	\newcommand{\mrP}{\mathrm{P}}
\newcommand{\mrQ}{\mathrm{Q}}	\newcommand{\mrR}{\mathrm{R}}
\newcommand{\mrS}{\mathrm{S}}	\newcommand{\mrT}{\mathrm{T}}
\newcommand{\mrU}{\mathrm{U}}	\newcommand{\mrV}{\mathrm{V}}
\newcommand{\mrW}{\mathrm{W}}	\newcommand{\mrX}{\mathrm{X}}
\newcommand{\mrY}{\mathrm{Y}}	\newcommand{\mrZ}{\mathrm{Z}}
%Small Letters
\newcommand{\mra}{\mathrm{a}}	\newcommand{\mrb}{\mathrm{b}}
\newcommand{\mrc}{\mathrm{c}}	\newcommand{\mrd}{\mathrm{d}}
\newcommand{\mre}{\mathrm{e}}	\newcommand{\mrf}{\mathrm{f}}
\newcommand{\mrg}{\mathrm{g}}	\newcommand{\mrh}{\mathrm{h}}
\newcommand{\mri}{\mathrm{i}}	\newcommand{\mrj}{\mathrm{j}}
\newcommand{\mrk}{\mathrm{k}}	\newcommand{\mrl}{\mathrm{l}}
\newcommand{\mrm}{\mathrm{m}}	\newcommand{\mrn}{\mathrm{n}}
\newcommand{\mro}{\mathrm{o}}	\newcommand{\mrp}{\mathrm{p}}
\newcommand{\mrq}{\mathrm{q}}	\newcommand{\mrr}{\mathrm{r}}
\newcommand{\mrs}{\mathrm{s}}	\newcommand{\mrt}{\mathrm{t}}
\newcommand{\mru}{\mathrm{u}}	\newcommand{\mrv}{\mathrm{v}}
\newcommand{\mrw}{\mathrm{w}}	\newcommand{\mrx}{\mathrm{x}}
\newcommand{\mry}{\mathrm{y}}	\newcommand{\mrz}{\mathrm{z}}

\ctexset{
    today=old
}

\title{\Huge{Linear Algebra}\\ Assignment 8}
\author{\Large{110307039\ 財管四\ 黃柏維}}
\date{\today}

\begin{document}

\maketitle
\cleardoublepage

\qs{}{(CH6.4 Q12) Find the matrix \( A' \) for \( T \) relative to the basis \( B' \), where \[ \begin{align*}
             & T: \RR[3] \to \RR[3],                          \\
             & T(x, y, z) = (x, x + 2y, x + y + 3z),          \\
             & B' = \{ (1, - 1, 0), (0, 0, 1), (0, 1, - 1) \}
        \end{align*}
    \]}

\solve{
    \begin{align*}
        T(1, - 1, 0) & = (1, - 1, 0) \\
        T(0, 0, 1)   & = (0, - 2, 3) \\
        T(0, 1, - 1) & = (0, 2, - 2)
    \end{align*}
    Thus the matrix \( A' \) is \[ A' = \begin{bmatrix}
            1 & - 1 & 0  \\
            0 & - 2 & 3  \\
            0 & 2   & -2
        \end{bmatrix} \]
}

\qs{}{(CH6.4 Q14) Let \( B = \{ (1, 1), ( - 2, 3) \}, B' = \{ (1, - 1), (0, 1) \}, and \left[ \bmv \right]_{B'} = \begin{matrix}
        1 & - 3
    \end{matrix}^T \), and let \[
        A = \begin{bmatrix}
            3 & 2 \\
            0 & 4
        \end{bmatrix}
    \]
    be the matrix for \( T: \RR[2] \to \RR[2] \) relative to \( B. \)

    \begin{enumerate}[label=(\alph*)]
        \item Find the transition matrix \( P \) from \( B' \) to \( B. \)
        \item Use the matrices \( P \) and \( A \) to find \( \left[ \bmv \right]_B \) and \(
              \left[ T(\bmv) \right]_{B} \), where \( \left[ \bmv\right]_{B'} = \begin{matrix}
                  - 1 & 2
              \end{matrix}^T \).
        \item Find \( P^{-1} \) and \( A' \) (the matrix for \( T \) relative to \( B' \)).
        \item Find \( \begin{bmatrix} T(\bmv) \end{bmatrix}_{B'} \) two ways.
    \end{enumerate}
}

\solve{
    \begin{enumerate}[label=(\alph*)]
        \item \[
                  P \;=\;
                  \begin{bmatrix}
                      \dfrac15  & \dfrac25 \\[4pt]
                      -\dfrac25 & \dfrac15
                  \end{bmatrix},
                  \qquad
                  [\mathbf{x}]_{B}=P\,[\mathbf{x}]_{B'}.
              \]

        \item \([\mathbf{v}]_{B'}=\begin{bmatrix}-1\\2\end{bmatrix}\):

              \[
                  [\mathbf{v}]_{B}
                  =P\,[\mathbf{v}]_{B'}
                  =
                  \begin{bmatrix}\tfrac35\\[4pt]\tfrac45\end{bmatrix},
                  \qquad
                  [T(\mathbf{v})]_{B}
                  =A\,[\mathbf{v}]_{B}
                  =
                  \begin{bmatrix}\tfrac{17}{5}\\[4pt]\tfrac{16}{5}\end{bmatrix}.
              \]

        \item \[
                  P^{-1}
                  =
                  \begin{bmatrix}
                      1 & -2  \\
                      2 & \;1
                  \end{bmatrix},
                  \qquad
                  A'
                  =P^{-1} A P
                  =
                  \begin{bmatrix}
                      3  & 0 \\
                      -2 & 4
                  \end{bmatrix}.
              \]

        \item      \[
                  \begin{aligned}
                      [\;T(\mathbf{v})\;]_{B'}
                       & =A'\,[\mathbf{v}]_{B'}
                      =\begin{bmatrix}3&0\\-2&4\end{bmatrix}
                      \begin{bmatrix}-1\\2\end{bmatrix}
                      =\begin{bmatrix}-3\\10\end{bmatrix}, \\[6pt]
                      [\;T(\mathbf{v})\;]_{B'}
                       & =P^{-1}[\;T(\mathbf{v})\;]_{B}
                      =\begin{bmatrix}1&-2\\2&1\end{bmatrix}
                      \begin{bmatrix}\tfrac{17}{5}\\[4pt]\tfrac{16}{5}\end{bmatrix}
                      =\begin{bmatrix}-3\\10\end{bmatrix}.
                  \end{aligned}
              \]
    \end{enumerate}
}

\qs{}{(CH6.4 Q22) Use the matrix \( P \) to show that the matrices \( A \) and \( A' \) are similar.
    \[
        P = \begin{bmatrix}
            1 & 1 & 1 \\
            0 & 1 & 1 \\
            0 & 0 & 1
        \end{bmatrix}, \quad A = \begin{bmatrix}
            5 & 0 & 0 \\
            0 & 3 & 0 \\
            0 & 0 & 1
        \end{bmatrix}, \quad A' = \begin{bmatrix}
            5 & 2 & 2 \\
            0 & 3 & 2 \\
            0 & 0 & 1
        \end{bmatrix}
    \]
}

\solve{
    To prove that \(A\) and \(A'\) are similar we must show that
    \(
    A' = P^{-1}AP.
    \)

    \[
        P
        =\begin{bmatrix}
            1 & 1 & 1 \\
            0 & 1 & 1 \\
            0 & 0 & 1
        \end{bmatrix},
        \qquad
        A
        =\begin{bmatrix}
            5 & 0 & 0 \\
            0 & 3 & 0 \\
            0 & 0 & 1
        \end{bmatrix}.
    \]

    \textbf{1.  Compute \(P^{-1}\).}

    Because \(P\) is unit upper-triangular, its inverse is also unit
    upper-triangular:

    \[
        P^{-1}
        =\begin{bmatrix}
            1 & -1 & 0  \\
            0 & 1  & -1 \\
            0 & 0  & 1
        \end{bmatrix}.
    \]

    \textbf{2.  Evaluate \(P^{-1}AP\).}

    \[
        \begin{aligned}
            P^{-1}A
             & =\begin{bmatrix}
                    1 & -1 & 0  \\
                    0 & 1  & -1 \\
                    0 & 0  & 1
                \end{bmatrix}
            \begin{bmatrix}
                5 & 0 & 0 \\
                0 & 3 & 0 \\
                0 & 0 & 1
            \end{bmatrix}
            =\begin{bmatrix}
                 5 & -3 & 0  \\
                 0 & 3  & -1 \\
                 0 & 0  & 1
             \end{bmatrix},    \\[10pt]
            P^{-1}AP
             & =\begin{bmatrix}
                    5 & -3 & 0  \\
                    0 & 3  & -1 \\
                    0 & 0  & 1
                \end{bmatrix}
            \begin{bmatrix}
                1 & 1 & 1 \\
                0 & 1 & 1 \\
                0 & 0 & 1
            \end{bmatrix}
            =\begin{bmatrix}
                 5 & 2 & 2 \\
                 0 & 3 & 2 \\
                 0 & 0 & 1
             \end{bmatrix}
            =A'.
        \end{aligned}
    \]

    \textbf{3.  Conclusion.}

    Since \(A' = P^{-1}AP\) (equivalently \(A = PAP^{-1}\)), the matrices \(A\) and
    \(A'\) are similar. Therefore they represent the same linear transformation
    with respect to two different bases.

}

\qs{}{(CH7.1 Q6) Verify that \( \lambda_i \) is an eigenvalue of \( A \) and that \( \bmx_i \) is a corresponding eigenvector.
    \[
        A = \begin{bmatrix}
            4 & - 1 & 3 \\
            0 & 2   & 1 \\
            0 & 0   & 3
        \end{bmatrix}, \quad \begin{matrix}
            \lambda_1 = 4, \quad \bmx_1 = ( 1, 0, 0) \\
            \lambda_2 = 2, \quad \bmx_2 = ( 1, 2, 0) \\
            \lambda_3 = 3, \quad \bmx_3 = ( - 2, 1, 1)
        \end{matrix}
    \]}

\solve{
    \textbf{Verification that each \(\lambda_i\) is an eigenvalue of \(A\) with eigenvector \(\mathbf{x}_i\).}

    \[
        A
        =\begin{bmatrix}
            4 & -1 & 3 \\
            0 & 2  & 1 \\
            0 & 0  & 3
        \end{bmatrix}
        \qquad
        \mathbf{x}_1=\begin{bmatrix}1\\0\\0\end{bmatrix},
        \;
        \mathbf{x}_2=\begin{bmatrix}1\\2\\0\end{bmatrix},
        \;
        \mathbf{x}_3=\begin{bmatrix}-2\\1\\1\end{bmatrix}.
    \]

    \begin{enumerate}[label=(\alph*)]
        \item \emph{Eigenpair} \(\lambda_1 = 4,\; \mathbf{x}_1\):

              \[
                  A\mathbf{x}_1
                  =\begin{bmatrix}
                      4 & -1 & 3 \\
                      0 & 2  & 1 \\
                      0 & 0  & 3
                  \end{bmatrix}
                  \begin{bmatrix}1\\0\\0\end{bmatrix}
                  =\begin{bmatrix}4\\0\\0\end{bmatrix}
                  =4\begin{bmatrix}1\\0\\0\end{bmatrix}
                  =\lambda_1\mathbf{x}_1.
              \]

        \item \emph{Eigenpair} \(\lambda_2 = 2,\; \mathbf{x}_2\):

              \[
                  A\mathbf{x}_2
                  =\begin{bmatrix}
                      4 & -1 & 3 \\
                      0 & 2  & 1 \\
                      0 & 0  & 3
                  \end{bmatrix}
                  \begin{bmatrix}1\\2\\0\end{bmatrix}
                  =\begin{bmatrix}4-2\\4\\0\end{bmatrix}
                  =\begin{bmatrix}2\\4\\0\end{bmatrix}
                  =2\begin{bmatrix}1\\2\\0\end{bmatrix}
                  =\lambda_2\mathbf{x}_2.
              \]

        \item \emph{Eigenpair} \(\lambda_3 = 3,\; \mathbf{x}_3\):

              \[
                  A\mathbf{x}_3
                  =\begin{bmatrix}
                      4 & -1 & 3 \\
                      0 & 2  & 1 \\
                      0 & 0  & 3
                  \end{bmatrix}
                  \begin{bmatrix}-2\\1\\1\end{bmatrix}
                  =\begin{bmatrix}-8-1+3\\2+1\\3\end{bmatrix}
                  =\begin{bmatrix}-6\\3\\3\end{bmatrix}
                  =3\begin{bmatrix}-2\\1\\1\end{bmatrix}
                  =\lambda_3\mathbf{x}_3.
              \]
    \end{enumerate}

    \[
        A\mathbf{x}_i=\lambda_i\mathbf{x}_i\quad\text{for }i=1,2,3.
    \]

    Hence each \(\lambda_i\) is an eigenvalue of \(A\) and each \(\mathbf{x}_i\) is
    a corresponding eigenvector.

}

\qs{}{(CH7.1 Q12) Determine whether \( \bmx \) is an eigenvector of \( A. \)
    \[
        A = \begin{bmatrix}
            1   & 0   & 5   \\
            - 2 & 0   & - 2 \\
            3   & - 3 & 1
        \end{bmatrix}
    \]
    \begin{enumerate}[label=(\alph*)]
        \item \( \bmx = (1, 1, 0) \)
        \item \( \bmx = ( - 5, 2, 1) \)
        \item \( \bmx = ( 0, 0, 0) \)
        \item \( \bmx = ( 2 \sqrt{6} - 3, - 2 \sqrt{6} + 6, 3) \)
    \end{enumerate}

}

\solve{
    \textbf{Test for each candidate vector \(\mathbf{x}\) whether \(A\mathbf{x}=\lambda\mathbf{x}\) for some scalar \(\lambda\).}

    \[
        A=\begin{bmatrix}
            1  & 0  & 5  \\
            -2 & 0  & -2 \\
            3  & -3 & 1
        \end{bmatrix}
    \]

    \begin{enumerate}[label=(\alph*)]
        \item \(\mathbf{x}=(1,1,0)^{\mathsf T}\)

              \[
                  A\mathbf{x}
                  =\begin{bmatrix}
                      1 \\-2\\0
                  \end{bmatrix},\qquad
                  \frac{1}{1}\neq\frac{-2}{1},\;
                  \mathbf{x}_3=0.
              \]
              The three coordinates are not in a common ratio, so no scalar \(\lambda\)
              satisfies \(A\mathbf{x}=\lambda\mathbf{x}\). \( \textbf{Not an eigenvector.} \)

        \item \(\mathbf{x}=(-5,2,1)^{\mathsf T}\)

              \[
                  A\mathbf{x}
                  =\begin{bmatrix}
                      0 \\ 8\\ -20
                  \end{bmatrix},\qquad
                  \frac{0}{-5}\neq\frac{8}{2}.
              \]
              No common ratio exists. \( \textbf{Not an eigenvector.} \)

        \item \(\mathbf{x}=\mathbf{0}\)

              By definition the zero vector can \emph{never} be an eigenvector (eigenvectors
              must be non-zero). \( \textbf{Not an eigenvector.} \)

        \item \(\mathbf{x}=(2\sqrt{6}-3,\,-2\sqrt{6}+6,\,3)^{\mathsf T}\)

              \[
                  A\mathbf{x}
                  =\begin{bmatrix}
                      2\sqrt{6}+12 \\
                      -4\sqrt{6}   \\
                      12\sqrt{6}-24
                  \end{bmatrix},\qquad
                  \frac{2\sqrt{6}+12}{2\sqrt{6}-3}\neq
                  \frac{-4\sqrt{6}}{-2\sqrt{6}+6}.
              \]
              Again the coordinates fail to share a common ratio. \( \textbf{Not an
                  eigenvector.} \)
    \end{enumerate}
}

\qs{}{(CH7.1 Q26) Find (a) the characteristic equation and (b) the eigenvalues of (and corresponding eigenvectors) of the matrix
    \[
        A = \begin{bmatrix}
            1           & - \frac{3}{2} & \frac{5}{2} \\
            - 2         & \frac{13}{2}  & - 10        \\
            \frac{3}{2} & - \frac{9}{2} & 8
        \end{bmatrix}
    \]
}

\solve{
    \[
        A \;=\;
        \begin{bmatrix}
            1        & -\tfrac32     & \tfrac52 \\
            -2       & \tfrac{13}{2} & -10      \\
            \tfrac32 & -\tfrac92     & 8
        \end{bmatrix}
    \]

    \begin{enumerate}[label=(\alph*)]
        \item \textbf{Characteristic equation}

              \[
                  \det\!\bigl(A-\lambda I\bigr)
                  =\lambda^{3}-\frac{31}{2}\lambda^{2}
                  +\frac{59}{4}\lambda-\frac{29}{8}
                  =\frac18\,(2\lambda-1)^{2}(2\lambda-29)=0.
              \]

        \item \textbf{Eigenvalues and eigenvectors}

              \[
                  \begin{aligned}
                      \lambda_1 & =\frac12
                                &                            & \text{(algebraic multiplicity $2$)} \\[4pt]
                                & \quad\text{eigenvectors: }
                      \operatorname{span}\!\left\{
                      \begin{bmatrix}3\\1\\0\end{bmatrix},
                      \begin{bmatrix}-5\\0\\1\end{bmatrix}
                      \right\};                                                                    \\[12pt]
                      \lambda_2 & =\frac{29}{2}
                                &                            & \text{(algebraic multiplicity $1$)} \\[4pt]
                                & \quad\text{eigenvectors: }
                      \operatorname{span}\!\left\{
                      \begin{bmatrix}1\\-4\\3\end{bmatrix}
                      \right\}.
                  \end{aligned}
              \]
    \end{enumerate}

}

\qs{}{(CH7.2 Q14) Find (if possible) a nonsingular matrix \( P \) such that \( P^{-1}AP \) is diagonal. Verify that \( P^{-1}AP \) is a diagonal matrix with the eigenvalues on the main diagonal.
    \[
        A = \begin{bmatrix}
            2 & 0 & 0 \\
            4 & 4 & 0 \\
            0 & 4 & 4
        \end{bmatrix}
    \]
}

\solve{
    \[
        A=\begin{bmatrix}
            2 & 0 & 0 \\
            4 & 4 & 0 \\
            0 & 4 & 4
        \end{bmatrix}
    \]

    \begin{enumerate}[label=(\alph*)]
        \item \textbf{Characteristic polynomial}

              \[
                  \det(A-\lambda I)=
                  \det\!
                  \begin{bmatrix}
                      2-\lambda & 0         & 0         \\
                      4         & 4-\lambda & 0         \\
                      0         & 4         & 4-\lambda
                  \end{bmatrix}
                  =(2-\lambda)\bigl((4-\lambda)^2\bigr)
                  =(2-\lambda)(4-\lambda)^2.
              \]

              Hence
              \[
                  \lambda_1=2,\qquad \lambda_2=\lambda_3=4\;(\text{algebraic multiplicity }2).
              \]

        \item \textbf{Eigenvectors}

              \smallskip
              \[
                  \begin{aligned}
                      \lambda=2:\; &
                      (A-2I)\mathbf{x}=0
                      \;\Longrightarrow\;
                      \mathbf{x}=t\begin{bmatrix}1\\-2\\4\end{bmatrix},
                      \;\text{so }E_{2}=\operatorname{span}\!\left\{\begin{bmatrix}1\\-2\\4\end{bmatrix}\right\},\;
                      \dim E_{2}=1.  \\[6pt]
                      \lambda=4:\; &
                      (A-4I)\mathbf{x}=0
                      \;\Longrightarrow\;
                      \mathbf{x}=s\begin{bmatrix}0\\0\\1\end{bmatrix},
                      \;\text{so }E_{4}=\operatorname{span}\!\left\{\begin{bmatrix}0\\0\\1\end{bmatrix}\right\},\;
                      \dim E_{4}=1.
                  \end{aligned}
              \]

        \item \textbf{Diagonalizability test}

              To diagonalize \(A\) we need
              \[
                  \sum_{\lambda}\dim E_{\lambda}=n=3.
              \]
              Here \( \dim E_{2}+\dim E_{4}=1+1=2<3, \) so \(A\) does \emph{not} possess
              three linearly independent eigenvectors.

        \item \textbf{Conclusion}

              Because the geometric multiplicity of \(\lambda=4\) is \(1<2\) (its algebraic
              multiplicity), matrix \(A\) is \emph{not} diagonalizable. Consequently, there
              is no nonsingular matrix \(P\) for which \(P^{-1}AP\) is diagonal.
    \end{enumerate}
}

\qs{}{(CH7.2 Q20) Show that the matrix \( A \) is not diagonalizable, where \[
        A = \begin{bmatrix}
            3 & 2   & - 2 \\
            0 & - 2 & 3   \\
            0 & 0\, & - 2
        \end{bmatrix}
    \]
}

\solve{
    \section*{1.  Characteristic polynomial}

    Because \(A\) is upper-triangular, its eigenvalues are the diagonal entries.
    Formally,
    \[
        \chi_A(\lambda)=\det(A-\lambda I)
        =(3-\lambda)(-2-\lambda)^2.
    \]

    \[
        \lambda_1=3\;(\text{alg.\,mult. }1),\qquad
        \lambda_2=-2\;(\text{alg.\,mult. }2)
    \]

    \section*{2.  Eigenspaces}

    \subsection*{(a)  \(\lambda=3\)}

    \[
        A-3I=
        \begin{bmatrix}
            0 & 2  & -2 \\
            0 & -5 & 3  \\
            0 & 0  & -5
        \end{bmatrix},
        \qquad
        (A-3I)\mathbf{x}=0
        \Longrightarrow
        \mathbf{x}=t\begin{bmatrix}1\\0\\0\end{bmatrix}.
    \]

    \[
        E_{3}=\operatorname{span}\!\left\{\begin{bmatrix}1\\0\\0\end{bmatrix}\right\},
        \qquad
        \dim E_{3}=1.
    \]

    \subsection*{(b)  \(\lambda=-2\)}

    \[
        A+2I=
        \begin{bmatrix}
            5 & 2 & -2 \\
            0 & 0 & 3  \\
            0 & 0 & 0
        \end{bmatrix},
        \qquad
        (A+2I)\mathbf{x}=0
        \Longrightarrow
        \mathbf{x}=s\begin{bmatrix}2\\-5\\0\end{bmatrix}.
    \]

    \[
        E_{-2}=\operatorname{span}\!\left\{\begin{bmatrix}2\\-5\\0\end{bmatrix}\right\},
        \qquad
        \dim E_{-2}=1.
    \]

    \section*{3.  Diagonalizability test}

    A matrix of size \(n=3\) is diagonalizable iff the sum of the geometric
    multiplicities equals \(n\).

    \[
        \dim E_{3}+\dim E_{-2}=1+1=2<3.
    \]

    Hence \(A\) has only two linearly independent eigenvectors.

    \[
        \text{$A$ is \emph{not} diagonalizable}
    \]

    The failure arises because the geometric multiplicity of \(\lambda=-2\) (\(1\))
    is strictly less than its algebraic multiplicity (\(2\)).

}

\qs{}{(CH7.2 Q26) Find the eigenvalues of the matrix \( A \) and determine whether there is a sufficient number of eigenvalues to guarantee the matrix is diagonalizable by Theorem 7.6, where \[
        A = \begin{bmatrix}
            4 & 3 & - 2 \\
            0 & 1 & 1   \\
            0 & 0 & - 2
        \end{bmatrix},
    \] and Theorem 7.6 states that \[ \begin{align*}
             & \text{If an } n \times n \text{ matrix } A \text{ has } n \; \emph{distinct} \text{ eigenvalues, then the corresponding eigenvectors are linearly} \\
             & \text{independent and } A \text{ is diagonalizable.}
        \end{align*}
    \]
}

\solve{
    \begin{enumerate}[label=(\alph*)]
        \item \textbf{Eigenvalues}

              Because \(A\) is upper-triangular, its eigenvalues are the diagonal entries:

              \[
                  \lambda_1 = 4,\quad \lambda_2 = 1,\quad \lambda_3 = -2.
              \]

        \item \textbf{Test for diagonalizability via Theorem 7.6}

              The matrix is \(3\times 3\) (\(n = 3\)) and has \(n\) \emph{distinct}
              eigenvalues \(\{4,1,-2\}\).

              \[   \begin{align*}
                       & \text{Therefore, by Theorem 7.6, the three corresponding eigenvectors are linearly independent, } \\
                       & \text{and } A \text{ is diagonalizable.}
                  \end{align*}
              \]
    \end{enumerate}

}

\qs{}{(CH7.2 Q30) Find  a basis \( B \) for the domain of \( T \) such that the matrix for \( T \) relative to \( B \) is diagonal.
    \[
        T: P_2 \to P_2: \\
        T(c + bx + ax^2) = (3c + a) + (2b + 3a)x + ax^2
    \]
}

\solve{
%----------------------------------------
%  Diagonalizing the operator T on P_2
%----------------------------------------

\[
    T\!\bigl(c+bx+ax^{2}\bigr)
    =(3c+a)\;+\;(2b+3a)x\;+\;a\,x^{2},
    \qquad P_{2}\equiv\{\text{polynomials of degree }\le 2\}.
\]

\subsection*{1.  Matrix of \(T\) in the standard basis \(S=\{1,x,x^{2}\}\)}

Writing a polynomial \(p=c+bx+ax^{2}\) as the column vector \(
[c\;b\;a]^{\mathsf T}, \) the action of \(T\) is

\[
    \begin{bmatrix}c\\ b\\ a\end{bmatrix}
    \;\xmapsto{\,T\,}\;
    \begin{bmatrix}
        3c+a \\ 2b+3a\\ a
    \end{bmatrix}
    \;=\;
    \underbrace{\begin{bmatrix}
            3 & 0 & 1 \\
            0 & 2 & 3 \\
            0 & 0 & 1
        \end{bmatrix}}_{A=[T]_{S}}
    \begin{bmatrix}
        c \\ b\\ a
    \end{bmatrix}.
\]

\subsection*{2.  Eigenvalues of \(A\)}

Because \(A\) is upper-triangular, its eigenvalues are the diagonal entries:

\[
    \lambda_{1}=3,\qquad
    \lambda_{2}=2,\qquad
    \lambda_{3}=1
    \quad(\text{all distinct}).
\]

\subsection*{3.  Eigenvectors}

\[
    \begin{aligned}
        \lambda_{1}=3: & \;
        (A-3I)
        =\begin{bmatrix}
             0 & 0  & 1  \\
             0 & -1 & 3  \\
             0 & 0  & -2
         \end{bmatrix}
        \;\;\Longrightarrow\;\;
        \mathbf{v}_{1}
        =\begin{bmatrix}1\\0\\0\end{bmatrix}
        \;\;(p_{1}=1).      \\[6pt]
        %
        \lambda_{2}=2: & \;
        (A-2I)
        =\begin{bmatrix}
             1 & 0 & 1  \\
             0 & 0 & 3  \\
             0 & 0 & -1
         \end{bmatrix}
        \;\;\Longrightarrow\;\;
        \mathbf{v}_{2}
        =\begin{bmatrix}0\\1\\0\end{bmatrix}
        \;\;(p_{2}=x).      \\[6pt]
        %
        \lambda_{3}=1: & \;
        (A-I)
        =\begin{bmatrix}
             2 & 0 & 1 \\
             0 & 1 & 3 \\
             0 & 0 & 0
         \end{bmatrix}
        \;\;\Longrightarrow\;\;
        \mathbf{v}_{3}
        =\begin{bmatrix}-1\\-6\\2\end{bmatrix}
        \;\;(p_{3}=2x^{2}-6x-1).
    \end{aligned}
\]

\subsection*{4.  A diagonalizing basis and the diagonal matrix}

Set
\[
    B=\{\,p_{1}=1,\;p_{2}=x,\;p_{3}=2x^{2}-6x-1\,\}.
\]
Since the three eigenvalues are distinct,
\(\{\mathbf{v}_{1},\mathbf{v}_{2},\mathbf{v}_{3}\}\) is linearly independent,
so \(B\) is a basis of \(P_{2}\).

With \(P=[\,\mathbf{v}_{1}\;\mathbf{v}_{2}\;\mathbf{v}_{3}\,]\) the
change-of-basis matrix from \(B\) to \(S\),

\[
    P^{-1}AP
    =\operatorname{diag}(3,2,1)
    =\begin{bmatrix}
        3 & 0 & 0 \\
        0 & 2 & 0 \\
        0 & 0 & 1
    \end{bmatrix}.
\]

\[
    [T]_{B}=\operatorname{diag}(3,2,1)
\]

Hence \(T\) is diagonal in the eigenbasis \(B\).

}

\end{document}