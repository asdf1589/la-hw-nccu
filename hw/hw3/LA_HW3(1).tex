\documentclass{report}

\usepackage{ctex}

\usepackage[tmargin=2cm,rmargin=1in,lmargin=1in,margin=0.85in,bmargin=2cm,footskip=.2in]{geometry}
\usepackage{amsmath,amsfonts,amsthm,amssymb,mathtools}
\usepackage[varbb]{newpxmath}
\usepackage{xfrac}
\usepackage[makeroom]{cancel}
\usepackage{mathtools}
\usepackage{bookmark}
\usepackage{enumitem}
\usepackage{hyperref,theoremref}
\hypersetup{
	pdftitle={assignment},
	colorlinks=true, linkcolor=doc!90,
	bookmarksnumbered=true,
	bookmarksopen=true
}
\usepackage[most,many,breakable]{tcolorbox}
\usepackage{xcolor}
\usepackage{varwidth}
\usepackage{varwidth}
\usepackage{etoolbox}
%\usepackage{authblk}
\usepackage{nameref}
\usepackage{multicol,array}
\usepackage[ruled,vlined,linesnumbered]{algorithm2e}
\usepackage{comment} % enables the use of multi-line comments (\ifx \fi) 
\usepackage{import}
\usepackage{xifthen}
\usepackage{pdfpages}
\usepackage{transparent}
\usepackage{chngcntr}
\usepackage{tikz}
\usepackage{titletoc}

\newcommand\mycommfont[1]{\footnotesize\ttfamily\textcolor{blue}{#1}}
\SetCommentSty{mycommfont}
\newcommand{\incfig}[1]{%
    \def\svgwidth{\columnwidth}
    \import{./figures/}{#1.pdf_tex}
}

\usepackage{tikzsymbols}
\tikzset{
	symbol/.style={
			draw=none,
			every to/.append style={
					edge node={node [sloped, allow upside down, auto=false]{$#1$}}}
		}
}
\tikzstyle{c} = [circle,fill=black,scale=0.5]
\tikzstyle{b} = [draw, thick, black, -]
\tikzset{
    vertex/.style={
        circle,
        draw,
        minimum size=6mm,
        inner sep=0pt
    }
}
\renewcommand\qedsymbol{$\Laughey$}

%\usepackage{import}
%\usepackage{xifthen}
%\usepackage{pdfpages}
%\usepackage{transparent}


%%%%%%%%%%%%%%%%%%%%%%%%%%%%%%
% SELF MADE COLORS
%%%%%%%%%%%%%%%%%%%%%%%%%%%%%%

\definecolor{doc}{RGB}{0,60,110}
\definecolor{myg}{RGB}{56, 140, 70}
\definecolor{myb}{RGB}{45, 111, 177}
\definecolor{myr}{RGB}{199, 68, 64}
\definecolor{mytheorembg}{HTML}{F2F2F9}
\definecolor{mytheoremfr}{HTML}{00007B}
\definecolor{mylemmabg}{HTML}{FFFAF8}
\definecolor{mylemmafr}{HTML}{983b0f}
\definecolor{mypropbg}{HTML}{f2fbfc}
\definecolor{mypropfr}{HTML}{191971}
\definecolor{myexamplebg}{HTML}{F2FBF8}
\definecolor{myexamplefr}{HTML}{88D6D1}
\definecolor{myexampleti}{HTML}{2A7F7F}
\definecolor{mydefinitbg}{HTML}{E5E5FF}
\definecolor{mydefinitfr}{HTML}{3F3FA3}
\definecolor{notesgreen}{RGB}{0,162,0}
\definecolor{myp}{RGB}{197, 92, 212}
\definecolor{mygr}{HTML}{2C3338}
\definecolor{myred}{RGB}{127,0,0}
\definecolor{myyellow}{RGB}{169,121,69}
\definecolor{myexercisebg}{HTML}{F2FBF8}
\definecolor{myexercisefg}{HTML}{88D6D1}

%%%%%%%%%%%%%%%%%%%%%%%%%%%%
% TCOLORBOX SETUPS
%%%%%%%%%%%%%%%%%%%%%%%%%%%%

\setlength{\parindent}{1cm}
%================================
% THEOREM BOX
%================================

\tcbuselibrary{theorems,skins,hooks}
\newtcbtheorem[number within=section]{Theorem}{Theorem}
{%
	enhanced,
	breakable,
	colback = mytheorembg,
	frame hidden,
	boxrule = 0sp,
	borderline west = {2pt}{0pt}{mytheoremfr},
	sharp corners,
	detach title,
	before upper = \tcbtitle\par\smallskip,
	coltitle = mytheoremfr,
	fonttitle = \bfseries\sffamily,
	description font = \mdseries,
	separator sign none,
	segmentation style={solid, mytheoremfr},
}
{th}

\tcbuselibrary{theorems,skins,hooks}
\newtcbtheorem[number within=chapter]{theorem}{Theorem}
{%
	enhanced,
	breakable,
	colback = mytheorembg,
	frame hidden,
	boxrule = 0sp,
	borderline west = {2pt}{0pt}{mytheoremfr},
	sharp corners,
	detach title,
	before upper = \tcbtitle\par\smallskip,
	coltitle = mytheoremfr,
	fonttitle = \bfseries\sffamily,
	description font = \mdseries,
	separator sign none,
	segmentation style={solid, mytheoremfr},
}
{th}


\tcbuselibrary{theorems,skins,hooks}
\newtcolorbox{Theoremcon}
{%
	enhanced
	,breakable
	,colback = mytheorembg
	,frame hidden
	,boxrule = 0sp
	,borderline west = {2pt}{0pt}{mytheoremfr}
	,sharp corners
	,description font = \mdseries
	,separator sign none
}

%================================
% Corollery
%================================
\tcbuselibrary{theorems,skins,hooks}
\newtcbtheorem[number within=section]{Corollary}{Corollary}
{%
	enhanced
	,breakable
	,colback = myp!10
	,frame hidden
	,boxrule = 0sp
	,borderline west = {2pt}{0pt}{myp!85!black}
	,sharp corners
	,detach title
	,before upper = \tcbtitle\par\smallskip
	,coltitle = myp!85!black
	,fonttitle = \bfseries\sffamily
	,description font = \mdseries
	,separator sign none
	,segmentation style={solid, myp!85!black}
}
{th}
\tcbuselibrary{theorems,skins,hooks}
\newtcbtheorem[number within=chapter]{corollary}{Corollary}
{%
	enhanced
	,breakable
	,colback = myp!10
	,frame hidden
	,boxrule = 0sp
	,borderline west = {2pt}{0pt}{myp!85!black}
	,sharp corners
	,detach title
	,before upper = \tcbtitle\par\smallskip
	,coltitle = myp!85!black
	,fonttitle = \bfseries\sffamily
	,description font = \mdseries
	,separator sign none
	,segmentation style={solid, myp!85!black}
}
{th}


%================================
% LEMMA
%================================

\tcbuselibrary{theorems,skins,hooks}
\newtcbtheorem[number within=section]{Lemma}{Lemma}
{%
	enhanced,
	breakable,
	colback = mylemmabg,
	frame hidden,
	boxrule = 0sp,
	borderline west = {2pt}{0pt}{mylemmafr},
	sharp corners,
	detach title,
	before upper = \tcbtitle\par\smallskip,
	coltitle = mylemmafr,
	fonttitle = \bfseries\sffamily,
	description font = \mdseries,
	separator sign none,
	segmentation style={solid, mylemmafr},
}
{th}

\tcbuselibrary{theorems,skins,hooks}
\newtcbtheorem[number within=chapter]{lemma}{lemma}
{%
	enhanced,
	breakable,
	colback = mylemmabg,
	frame hidden,
	boxrule = 0sp,
	borderline west = {2pt}{0pt}{mylemmafr},
	sharp corners,
	detach title,
	before upper = \tcbtitle\par\smallskip,
	coltitle = mylemmafr,
	fonttitle = \bfseries\sffamily,
	description font = \mdseries,
	separator sign none,
	segmentation style={solid, mylemmafr},
}
{th}

%================================
% Exercise
%================================

\tcbuselibrary{theorems,skins,hooks}
\newtcbtheorem[number within=section]{Exercise}{Exercise}
{%
	enhanced,
	breakable,
	colback = myexercisebg,
	frame hidden,
	boxrule = 0sp,
	borderline west = {2pt}{0pt}{myexercisefg},
	sharp corners,
	detach title,
	before upper = \tcbtitle\par\smallskip,
	coltitle = myexercisefg,
	fonttitle = \bfseries\sffamily,
	description font = \mdseries,
	separator sign none,
	segmentation style={solid, myexercisefg},
}
{th}

\tcbuselibrary{theorems,skins,hooks}
\newtcbtheorem[number within=chapter]{exercise}{Exercise}
{%
	enhanced,
	breakable,
	colback = myexercisebg,
	frame hidden,
	boxrule = 0sp,
	borderline west = {2pt}{0pt}{myexercisefg},
	sharp corners,
	detach title,
	before upper = \tcbtitle\par\smallskip,
	coltitle = myexercisefg,
	fonttitle = \bfseries\sffamily,
	description font = \mdseries,
	separator sign none,
	segmentation style={solid, myexercisefg},
}
{th}


%================================
% PROPOSITION
%================================

\tcbuselibrary{theorems,skins,hooks}
\newtcbtheorem[number within=section]{Prop}{Proposition}
{%
	enhanced,
	breakable,
	colback = mypropbg,
	frame hidden,
	boxrule = 0sp,
	borderline west = {2pt}{0pt}{mypropfr},
	sharp corners,
	detach title,
	before upper = \tcbtitle\par\smallskip,
	coltitle = mypropfr,
	fonttitle = \bfseries\sffamily,
	description font = \mdseries,
	separator sign none,
	segmentation style={solid, mypropfr},
}
{th}

\tcbuselibrary{theorems,skins,hooks}
\newtcbtheorem[number within=chapter]{prop}{Proposition}
{%
	enhanced,
	breakable,
	colback = mypropbg,
	frame hidden,
	boxrule = 0sp,
	borderline west = {2pt}{0pt}{mypropfr},
	sharp corners,
	detach title,
	before upper = \tcbtitle\par\smallskip,
	coltitle = mypropfr,
	fonttitle = \bfseries\sffamily,
	description font = \mdseries,
	separator sign none,
	segmentation style={solid, mypropfr},
}
{th}


%================================
% CLAIM
%================================

\tcbuselibrary{theorems,skins,hooks}
\newtcbtheorem[number within=section]{claim}{Claim}
{%
	enhanced
	,breakable
	,colback = myg!10
	,frame hidden
	,boxrule = 0sp
	,borderline west = {2pt}{0pt}{myg}
	,sharp corners
	,detach title
	,before upper = \tcbtitle\par\smallskip
	,coltitle = myg!85!black
	,fonttitle = \bfseries\sffamily
	,description font = \mdseries
	,separator sign none
	,segmentation style={solid, myg!85!black}
}
{th}



%================================
% EXAMPLE BOX
%================================

\newtcbtheorem[number within=section]{Example}{Example}
{%
	colback = myexamplebg
	,breakable
	,colframe = myexamplefr
	,coltitle = myexampleti
	,boxrule = 1pt
	,sharp corners
	,detach title
	,before upper=\tcbtitle\par\smallskip
	,fonttitle = \bfseries
	,description font = \mdseries
	,separator sign none
	,description delimiters parenthesis
}
{ex}

\newtcbtheorem[number within=chapter]{example}{Example}
{%
	colback = myexamplebg
	,breakable
	,colframe = myexamplefr
	,coltitle = myexampleti
	,boxrule = 1pt
	,sharp corners
	,detach title
	,before upper=\tcbtitle\par\smallskip
	,fonttitle = \bfseries
	,description font = \mdseries
	,separator sign none
	,description delimiters parenthesis
}
{ex}

%================================
% DEFINITION BOX
%================================

\newtcbtheorem[number within=section]{Definition}{Definition}{enhanced,
	before skip=2mm,after skip=2mm, colback=red!5,colframe=red!80!black,boxrule=0.5mm,
	attach boxed title to top left={xshift=1cm,yshift*=1mm-\tcboxedtitleheight}, varwidth boxed title*=-3cm,
	boxed title style={frame code={
					\path[fill=tcbcolback]
					([yshift=-1mm,xshift=-1mm]frame.north west)
					arc[start angle=0,end angle=180,radius=1mm]
					([yshift=-1mm,xshift=1mm]frame.north east)
					arc[start angle=180,end angle=0,radius=1mm];
					\path[left color=tcbcolback!60!black,right color=tcbcolback!60!black,
						middle color=tcbcolback!80!black]
					([xshift=-2mm]frame.north west) -- ([xshift=2mm]frame.north east)
					[rounded corners=1mm]-- ([xshift=1mm,yshift=-1mm]frame.north east)
					-- (frame.south east) -- (frame.south west)
					-- ([xshift=-1mm,yshift=-1mm]frame.north west)
					[sharp corners]-- cycle;
				},interior engine=empty,
		},
	fonttitle=\bfseries,
	title={#2},#1}{def}
\newtcbtheorem[number within=chapter]{definition}{Definition}{enhanced,
	before skip=2mm,after skip=2mm, colback=red!5,colframe=red!80!black,boxrule=0.5mm,
	attach boxed title to top left={xshift=1cm,yshift*=1mm-\tcboxedtitleheight}, varwidth boxed title*=-3cm,
	boxed title style={frame code={
					\path[fill=tcbcolback]
					([yshift=-1mm,xshift=-1mm]frame.north west)
					arc[start angle=0,end angle=180,radius=1mm]
					([yshift=-1mm,xshift=1mm]frame.north east)
					arc[start angle=180,end angle=0,radius=1mm];
					\path[left color=tcbcolback!60!black,right color=tcbcolback!60!black,
						middle color=tcbcolback!80!black]
					([xshift=-2mm]frame.north west) -- ([xshift=2mm]frame.north east)
					[rounded corners=1mm]-- ([xshift=1mm,yshift=-1mm]frame.north east)
					-- (frame.south east) -- (frame.south west)
					-- ([xshift=-1mm,yshift=-1mm]frame.north west)
					[sharp corners]-- cycle;
				},interior engine=empty,
		},
	fonttitle=\bfseries,
	title={#2},#1}{def}


%================================
% EXERCISE BOX
%================================

\newcounter{questioncounter}
\counterwithin{questioncounter}{chapter}
% \counterwithin{questioncounter}{section}

\makeatletter
\newtcbtheorem[use counter=questioncounter]{question}{Question}{enhanced,
	breakable,
	colback=white,
	colframe=myb!80!black,
	attach boxed title to top left={yshift*=-\tcboxedtitleheight},
	fonttitle=\bfseries,
	title={#2},
	boxed title size=title,
	boxed title style={%
			sharp corners,
			rounded corners=northwest,
			colback=tcbcolframe,
			boxrule=0pt,
		},
	underlay boxed title={%
			\path[fill=tcbcolframe] (title.south west)--(title.south east)
			to[out=0, in=180] ([xshift=5mm]title.east)--
			(title.center-|frame.east)
			[rounded corners=\kvtcb@arc] |-
			(frame.north) -| cycle;
		},
	#1
}{def}
\makeatother

%================================
% SOLUTION BOX
%================================

\makeatletter
\newtcolorbox{solution}{enhanced,
	breakable,
	colback=white,
	colframe=myg!80!black,
	attach boxed title to top left={yshift*=-\tcboxedtitleheight},
	title=Solution,
	boxed title size=title,
	boxed title style={%
			sharp corners,
			rounded corners=northwest,
			colback=tcbcolframe,
			boxrule=0pt,
		},
	underlay boxed title={%
			\path[fill=tcbcolframe] (title.south west)--(title.south east)
			to[out=0, in=180] ([xshift=5mm]title.east)--
			(title.center-|frame.east)
			[rounded corners=\kvtcb@arc] |-
			(frame.north) -| cycle;
		},
}
\makeatother

%================================
% Question BOX
%================================

\makeatletter
\newtcbtheorem{qstion}{Question}{enhanced,
    breakable,
    colback=white,
    colframe=mygr,
    attach boxed title to top left={yshift*=-\tcboxedtitleheight},
    fonttitle=\bfseries,
    title={#2},
    boxed title size=title,
    boxed title style={%
            sharp corners,
            rounded corners=northwest,
            colback=tcbcolframe,
            boxrule=0pt
        },
    underlay boxed title={%
            \path[fill=tcbcolframe](title.south west)--(title.south east)
            to[out=0,in=180]([xshift=5mm]title.east)--
            (title.center-|frame.east)
            [rounded corners=\kvtcb@arc]|-
            (frame.north)-|cycle;
        },
    #1
}{def}
\makeatother

\newtcbtheorem[number within=chapter]{wconc}{Wrong Concept}{
	breakable,
	enhanced,
	colback=white,
	colframe=myr,
	arc=0pt,
	outer arc=0pt,
	fonttitle=\bfseries\sffamily\large,
	colbacktitle=myr,
	attach boxed title to top left={},
	boxed title style={
			enhanced,
			skin=enhancedfirst jigsaw,
			arc=3pt,
			bottom=0pt,
			interior style={fill=myr}
		},
	#1
}{def}


%================================
% NOTE BOX
%================================

\usetikzlibrary{arrows,calc,shadows.blur}
\tcbuselibrary{skins}
\newtcolorbox{note}[1][]{%
	enhanced jigsaw,
	colback=gray!20!white,%
	colframe=gray!80!black,
	size=small,
	boxrule=1pt,
	title=\textbf{Note:-},
	halign title=flush center,
	coltitle=black,
	breakable,
	drop shadow=black!50!white,
	attach boxed title to top left={xshift=1cm,yshift=-\tcboxedtitleheight/2,yshifttext=-\tcboxedtitleheight/2},
	minipage boxed title=1.5cm,
	boxed title style={%
			colback=white,
			size=fbox,
			boxrule=1pt,
			boxsep=2pt,
			underlay={%
					\coordinate (dotA) at ($(interior.west) + (-0.5pt,0)$);
					\coordinate (dotB) at ($(interior.east) + (0.5pt,0)$);
					\begin{scope}
						\clip (interior.north west) rectangle ([xshift=3ex]interior.east);
						\filldraw [white, blur shadow={shadow opacity=60, shadow yshift=-.75ex}, rounded corners=2pt] (interior.north west) rectangle (interior.south east);
					\end{scope}
					\begin{scope}[gray!80!black]
						\fill (dotA) circle (2pt);
						\fill (dotB) circle (2pt);
					\end{scope}
				},
		},
	#1,
}

%%%%%%%%%%%%%%%%%%%%%%%%%%%%%%
% SELF MADE COMMANDS
%%%%%%%%%%%%%%%%%%%%%%%%%%%%%%

\newcommand{\thm}[2]{\begin{Theorem}{#1}{}#2\end{Theorem}}
\newcommand{\cor}[2]{\begin{Corollary}{#1}{}#2\end{Corollary}}
\newcommand{\mlemma}[2]{\begin{Lemma}{#1}{}#2\end{Lemma}}
\newcommand{\mer}[2]{\begin{Exercise}{#1}{}#2\end{Exercise}}
\newcommand{\mprop}[2]{\begin{Prop}{#1}{}#2\end{Prop}}
\newcommand{\clm}[3]{\begin{claim}{#1}{#2}#3\end{claim}}
\newcommand{\wc}[2]{\begin{wconc}{#1}{}\setlength{\parindent}{1cm}#2\end{wconc}}
\newcommand{\thmcon}[1]{\begin{Theoremcon}{#1}\end{Theoremcon}}
\newcommand{\ex}[2]{\begin{Example}{#1}{}#2\end{Example}}
\newcommand{\dfn}[2]{\begin{Definition}[colbacktitle=red!75!black]{#1}{}#2\end{Definition}}
\newcommand{\dfnc}[2]{\begin{definition}[colbacktitle=red!75!black]{#1}{}#2\end{definition}}
\newcommand{\qs}[2]{\begin{question}{#1}{}#2\end{question}}
\newcommand{\pf}[2]{\begin{myproof}[#1]#2\end{myproof}}
\newcommand{\nt}[1]{\begin{note}#1\end{note}}

\newcommand*\circled[1]{\tikz[baseline=(char.base)]{
		\node[shape=circle,draw,inner sep=1pt] (char) {#1};}}
\newcommand\getcurrentref[1]{%
	\ifnumequal{\value{#1}}{0}
	{??}
	{\the\value{#1}}%
}
\newcommand{\getCurrentSectionNumber}{\getcurrentref{section}}
\newenvironment{myproof}[1][\proofname]{%
	\proof[\bfseries #1: ]%
}{\endproof}

\newcommand{\mclm}[2]{\begin{myclaim}[#1]#2\end{myclaim}}
\newenvironment{myclaim}[1][\claimname]{\proof[\bfseries #1: ]}{}
\newenvironment{iclaim}[1][\claimname]{\bfseries #1\mdseries:}{}
\newcommand{\iclm}[2]{\begin{iclaim}[#1]#2\end{iclaim}}

\newcounter{mylabelcounter}

\makeatletter
\newcommand{\setword}[2]{%
	\phantomsection
	#1\def\@currentlabel{\unexpanded{#1}}\label{#2}%
}
\makeatother

% deliminators
\DeclarePairedDelimiter{\abs}{\lvert}{\rvert}
\DeclarePairedDelimiter{\norm}{\lVert}{\rVert}

\DeclarePairedDelimiter{\ceil}{\lceil}{\rceil}
\DeclarePairedDelimiter{\floor}{\lfloor}{\rfloor}
\DeclarePairedDelimiter{\round}{\lfloor}{\rceil}

\newsavebox\diffdbox
\newcommand{\slantedromand}{{\mathpalette\makesl{d}}}
\newcommand{\makesl}[2]{%
\begingroup
\sbox{\diffdbox}{$\mathsurround=0pt#1\mathrm{#2}$}%
\pdfsave
\pdfsetmatrix{1 0 0.2 1}%
\rlap{\usebox{\diffdbox}}%
\pdfrestore
\hskip\wd\diffdbox
\endgroup
}
\newcommand{\dd}[1][]{\ensuremath{\mathop{}\!\ifstrempty{#1}{%
\slantedromand\@ifnextchar^{\hspace{0.2ex}}{\hspace{0.1ex}}}%
{\slantedromand\hspace{0.2ex}^{#1}}}}
\ProvideDocumentCommand\dv{o m g}{%
  \ensuremath{%
    \IfValueTF{#3}{%
      \IfNoValueTF{#1}{%
        \frac{\dd #2}{\dd #3}%
      }{%
        \frac{\dd^{#1} #2}{\dd #3^{#1}}%
      }%
    }{%
      \IfNoValueTF{#1}{%
        \frac{\dd}{\dd #2}%
      }{%
        \frac{\dd^{#1}}{\dd #2^{#1}}%
      }%
    }%
  }%
}
\providecommand*{\pdv}[3][]{\frac{\partial^{#1}#2}{\partial#3^{#1}}}
%  - others
\DeclareMathOperator{\Lap}{\mathcal{L}}
\DeclareMathOperator{\Var}{Var} % varience
\DeclareMathOperator{\Cov}{Cov} % covarience
\DeclareMathOperator{\E}{E} % expected

% Since the amsthm package isn't loaded

% I prefer the slanted \leq
\let\oldleq\leq % save them in case they're every wanted
\let\oldgeq\geq
\renewcommand{\leq}{\leqslant}
\renewcommand{\geq}{\geqslant}

%%%%%%%%%%%%%%%%%%%%%%%%%%%%%%%%%%%%%%%%%%%
% TABLE OF CONTENTS
%%%%%%%%%%%%%%%%%%%%%%%%%%%%%%%%%%%%%%%%%%%

\contentsmargin{0cm}
\titlecontents{chapter}[3.7pc]
{\addvspace{30pt}%
	\begin{tikzpicture}[remember picture, overlay]%
		\draw[fill=doc!60,draw=doc!60] (-7,-.1) rectangle (-0.9,.5);%
		\pgftext[left,x=-3.7cm,y=0.2cm]{\color{white}\Large\sc\bfseries Chapter\ \thecontentslabel};%
	\end{tikzpicture}\color{doc!60}\large\sc\bfseries}%
{}
{}
{\;\titlerule\;\large\sc\bfseries Page \thecontentspage
	\begin{tikzpicture}[remember picture, overlay]
		\draw[fill=doc!60,draw=doc!60] (2pt,0) rectangle (4,0.1pt);
	\end{tikzpicture}}%
\titlecontents{section}[3.7pc]
{\addvspace{2pt}}
{\contentslabel[\thecontentslabel]{2pc}}
{}
{\hfill\small \thecontentspage}
[]
\titlecontents*{subsection}[3.7pc]
{\addvspace{-1pt}\small}
{}
{}
{\ --- \small\thecontentspage}
[ \textbullet\ ][]

\makeatletter
\renewcommand{\tableofcontents}{%
	\chapter*{%
	  \vspace*{-20\p@}%
	  \begin{tikzpicture}[remember picture, overlay]%
		  \pgftext[right,x=15cm,y=0.2cm]{\color{doc!60}\Huge\sc\bfseries \contentsname};%
		  \draw[fill=doc!60,draw=doc!60] (13,-.75) rectangle (20,1);%
		  \clip (13,-.75) rectangle (20,1);
		  \pgftext[right,x=15cm,y=0.2cm]{\color{white}\Huge\sc\bfseries \contentsname};%
	  \end{tikzpicture}}%
	\@starttoc{toc}}
\makeatother

\newcommand{\eps}{\epsilon}
\newcommand{\veps}{\varepsilon}
\newcommand{\Qed}{\begin{flushright}\qed\end{flushright}}

\newcommand{\parinn}{\setlength{\parindent}{1cm}}
\newcommand{\parinf}{\setlength{\parindent}{0cm}}

% \newcommand{\norm}{\|\cdot\|}
\newcommand{\inorm}{\norm_{\infty}}
\newcommand{\opensets}{\{V_{\alpha}\}_{\alpha\in I}}
\newcommand{\oset}{V_{\alpha}}
\newcommand{\opset}[1]{V_{\alpha_{#1}}}
\newcommand{\lub}{\text{lub}}
\newcommand{\del}[2]{\frac{\partial #1}{\partial #2}}
\newcommand{\Del}[3]{\frac{\partial^{#1} #2}{\partial^{#1} #3}}
\newcommand{\deld}[2]{\dfrac{\partial #1}{\partial #2}}
\newcommand{\Deld}[3]{\dfrac{\partial^{#1} #2}{\partial^{#1} #3}}
\newcommand{\der}[2]{\frac{\mathrm{d} #1}{\mathrm{d} #2}}
% \newcommand{\ddd}[3]{\frac{\mathrm{d}^{#3} #1}{\mathrm{d}^{#3} #2}}
\newcommand{\lm}{\lambda}
\newcommand{\uin}{\mathbin{\rotatebox[origin=c]{90}{$\in$}}}
\newcommand{\usubset}{\mathbin{\rotatebox[origin=c]{90}{$\subset$}}}
\newcommand{\lt}{\left}
\newcommand{\rt}{\right}
\newcommand{\bs}[1]{\boldsymbol{#1}}
\newcommand{\exs}{\exists}
\newcommand{\st}{\strut}
\newcommand{\dps}[1]{\displaystyle{#1}}
\newcommand{\id}{\text{id}}


\newcommand{\sol}{\setlength{\parindent}{0cm}\textbf{\textit{Solution:}}\setlength{\parindent}{1cm} }
\newcommand{\solve}[1]{\setlength{\parindent}{0cm}\textbf{\textit{Solution: }}\setlength{\parindent}{1cm}#1 \Qed}

% number sets
\newcommand{\RR}[1][]{\ensuremath{\ifstrempty{#1}{\mathbb{R}}{\mathbb{R}^{#1}}}}
\newcommand{\NN}[1][]{\ensuremath{\ifstrempty{#1}{\mathbb{N}}{\mathbb{N}^{#1}}}}
\newcommand{\ZZ}[1][]{\ensuremath{\ifstrempty{#1}{\mathbb{Z}}{\mathbb{Z}^{#1}}}}
\newcommand{\QQ}[1][]{\ensuremath{\ifstrempty{#1}{\mathbb{Q}}{\mathbb{Q}^{#1}}}}
\newcommand{\CC}[1][]{\ensuremath{\ifstrempty{#1}{\mathbb{C}}{\mathbb{C}^{#1}}}}
\newcommand{\PP}[1][]{\ensuremath{\ifstrempty{#1}{\mathbb{P}}{\mathbb{P}^{#1}}}}
\newcommand{\HH}[1][]{\ensuremath{\ifstrempty{#1}{\mathbb{H}}{\mathbb{H}^{#1}}}}
\newcommand{\FF}[1][]{\ensuremath{\ifstrempty{#1}{\mathbb{F}}{\mathbb{F}^{#1}}}}
% expected value
\newcommand{\EE}{\ensuremath{\mathbb{E}}}

%---------------------------------------
% BlackBoard Math Fonts :-
%---------------------------------------

%Captital Letters
\newcommand{\bbA}{\mathbb{A}}	\newcommand{\bbB}{\mathbb{B}}
\newcommand{\bbC}{\mathbb{C}}	\newcommand{\bbD}{\mathbb{D}}
\newcommand{\bbE}{\mathbb{E}}	\newcommand{\bbF}{\mathbb{F}}
\newcommand{\bbG}{\mathbb{G}}	\newcommand{\bbH}{\mathbb{H}}
\newcommand{\bbI}{\mathbb{I}}	\newcommand{\bbJ}{\mathbb{J}}
\newcommand{\bbK}{\mathbb{K}}	\newcommand{\bbL}{\mathbb{L}}
\newcommand{\bbM}{\mathbb{M}}	\newcommand{\bbN}{\mathbb{N}}
\newcommand{\bbO}{\mathbb{O}}	\newcommand{\bbP}{\mathbb{P}}
\newcommand{\bbQ}{\mathbb{Q}}	\newcommand{\bbR}{\mathbb{R}}
\newcommand{\bbS}{\mathbb{S}}	\newcommand{\bbT}{\mathbb{T}}
\newcommand{\bbU}{\mathbb{U}}	\newcommand{\bbV}{\mathbb{V}}
\newcommand{\bbW}{\mathbb{W}}	\newcommand{\bbX}{\mathbb{X}}
\newcommand{\bbY}{\mathbb{Y}}	\newcommand{\bbZ}{\mathbb{Z}}

%---------------------------------------
% MathCal Fonts :-
%---------------------------------------

%Captital Letters
\newcommand{\mcA}{\mathcal{A}}	\newcommand{\mcB}{\mathcal{B}}
\newcommand{\mcC}{\mathcal{C}}	\newcommand{\mcD}{\mathcal{D}}
\newcommand{\mcE}{\mathcal{E}}	\newcommand{\mcF}{\mathcal{F}}
\newcommand{\mcG}{\mathcal{G}}	\newcommand{\mcH}{\mathcal{H}}
\newcommand{\mcI}{\mathcal{I}}	\newcommand{\mcJ}{\mathcal{J}}
\newcommand{\mcK}{\mathcal{K}}	\newcommand{\mcL}{\mathcal{L}}
\newcommand{\mcM}{\mathcal{M}}	\newcommand{\mcN}{\mathcal{N}}
\newcommand{\mcO}{\mathcal{O}}	\newcommand{\mcP}{\mathcal{P}}
\newcommand{\mcQ}{\mathcal{Q}}	\newcommand{\mcR}{\mathcal{R}}
\newcommand{\mcS}{\mathcal{S}}	\newcommand{\mcT}{\mathcal{T}}
\newcommand{\mcU}{\mathcal{U}}	\newcommand{\mcV}{\mathcal{V}}
\newcommand{\mcW}{\mathcal{W}}	\newcommand{\mcX}{\mathcal{X}}
\newcommand{\mcY}{\mathcal{Y}}	\newcommand{\mcZ}{\mathcal{Z}}



%---------------------------------------
% Bold Math Fonts :-
%---------------------------------------

%Captital Letters
\newcommand{\bmA}{\boldsymbol{A}}	\newcommand{\bmB}{\boldsymbol{B}}
\newcommand{\bmC}{\boldsymbol{C}}	\newcommand{\bmD}{\boldsymbol{D}}
\newcommand{\bmE}{\boldsymbol{E}}	\newcommand{\bmF}{\boldsymbol{F}}
\newcommand{\bmG}{\boldsymbol{G}}	\newcommand{\bmH}{\boldsymbol{H}}
\newcommand{\bmI}{\boldsymbol{I}}	\newcommand{\bmJ}{\boldsymbol{J}}
\newcommand{\bmK}{\boldsymbol{K}}	\newcommand{\bmL}{\boldsymbol{L}}
\newcommand{\bmM}{\boldsymbol{M}}	\newcommand{\bmN}{\boldsymbol{N}}
\newcommand{\bmO}{\boldsymbol{O}}	\newcommand{\bmP}{\boldsymbol{P}}
\newcommand{\bmQ}{\boldsymbol{Q}}	\newcommand{\bmR}{\boldsymbol{R}}
\newcommand{\bmS}{\boldsymbol{S}}	\newcommand{\bmT}{\boldsymbol{T}}
\newcommand{\bmU}{\boldsymbol{U}}	\newcommand{\bmV}{\boldsymbol{V}}
\newcommand{\bmW}{\boldsymbol{W}}	\newcommand{\bmX}{\boldsymbol{X}}
\newcommand{\bmY}{\boldsymbol{Y}}	\newcommand{\bmZ}{\boldsymbol{Z}}
%Small Letters
\newcommand{\bma}{\boldsymbol{a}}	\newcommand{\bmb}{\boldsymbol{b}}
\newcommand{\bmc}{\boldsymbol{c}}	\newcommand{\bmd}{\boldsymbol{d}}
\newcommand{\bme}{\boldsymbol{e}}	\newcommand{\bmf}{\boldsymbol{f}}
\newcommand{\bmg}{\boldsymbol{g}}	\newcommand{\bmh}{\boldsymbol{h}}
\newcommand{\bmi}{\boldsymbol{i}}	\newcommand{\bmj}{\boldsymbol{j}}
\newcommand{\bmk}{\boldsymbol{k}}	\newcommand{\bml}{\boldsymbol{l}}
\newcommand{\bmm}{\boldsymbol{m}}	\newcommand{\bmn}{\boldsymbol{n}}
\newcommand{\bmo}{\boldsymbol{o}}	\newcommand{\bmp}{\boldsymbol{p}}
\newcommand{\bmq}{\boldsymbol{q}}	\newcommand{\bmr}{\boldsymbol{r}}
\newcommand{\bms}{\boldsymbol{s}}	\newcommand{\bmt}{\boldsymbol{t}}
\newcommand{\bmu}{\boldsymbol{u}}	\newcommand{\bmv}{\boldsymbol{v}}
\newcommand{\bmw}{\boldsymbol{w}}	\newcommand{\bmx}{\boldsymbol{x}}
\newcommand{\bmy}{\boldsymbol{y}}	\newcommand{\bmz}{\boldsymbol{z}}

%---------------------------------------
% Scr Math Fonts :-
%---------------------------------------

\newcommand{\sA}{{\mathscr{A}}}   \newcommand{\sB}{{\mathscr{B}}}
\newcommand{\sC}{{\mathscr{C}}}   \newcommand{\sD}{{\mathscr{D}}}
\newcommand{\sE}{{\mathscr{E}}}   \newcommand{\sF}{{\mathscr{F}}}
\newcommand{\sG}{{\mathscr{G}}}   \newcommand{\sH}{{\mathscr{H}}}
\newcommand{\sI}{{\mathscr{I}}}   \newcommand{\sJ}{{\mathscr{J}}}
\newcommand{\sK}{{\mathscr{K}}}   \newcommand{\sL}{{\mathscr{L}}}
\newcommand{\sM}{{\mathscr{M}}}   \newcommand{\sN}{{\mathscr{N}}}
\newcommand{\sO}{{\mathscr{O}}}   \newcommand{\sP}{{\mathscr{P}}}
\newcommand{\sQ}{{\mathscr{Q}}}   \newcommand{\sR}{{\mathscr{R}}}
\newcommand{\sS}{{\mathscr{S}}}   \newcommand{\sT}{{\mathscr{T}}}
\newcommand{\sU}{{\mathscr{U}}}   \newcommand{\sV}{{\mathscr{V}}}
\newcommand{\sW}{{\mathscr{W}}}   \newcommand{\sX}{{\mathscr{X}}}
\newcommand{\sY}{{\mathscr{Y}}}   \newcommand{\sZ}{{\mathscr{Z}}}


%---------------------------------------
% Math Fraktur Font
%---------------------------------------

%Captital Letters
\newcommand{\mfA}{\mathfrak{A}}	\newcommand{\mfB}{\mathfrak{B}}
\newcommand{\mfC}{\mathfrak{C}}	\newcommand{\mfD}{\mathfrak{D}}
\newcommand{\mfE}{\mathfrak{E}}	\newcommand{\mfF}{\mathfrak{F}}
\newcommand{\mfG}{\mathfrak{G}}	\newcommand{\mfH}{\mathfrak{H}}
\newcommand{\mfI}{\mathfrak{I}}	\newcommand{\mfJ}{\mathfrak{J}}
\newcommand{\mfK}{\mathfrak{K}}	\newcommand{\mfL}{\mathfrak{L}}
\newcommand{\mfM}{\mathfrak{M}}	\newcommand{\mfN}{\mathfrak{N}}
\newcommand{\mfO}{\mathfrak{O}}	\newcommand{\mfP}{\mathfrak{P}}
\newcommand{\mfQ}{\mathfrak{Q}}	\newcommand{\mfR}{\mathfrak{R}}
\newcommand{\mfS}{\mathfrak{S}}	\newcommand{\mfT}{\mathfrak{T}}
\newcommand{\mfU}{\mathfrak{U}}	\newcommand{\mfV}{\mathfrak{V}}
\newcommand{\mfW}{\mathfrak{W}}	\newcommand{\mfX}{\mathfrak{X}}
\newcommand{\mfY}{\mathfrak{Y}}	\newcommand{\mfZ}{\mathfrak{Z}}
%Small Letters
\newcommand{\mfa}{\mathfrak{a}}	\newcommand{\mfb}{\mathfrak{b}}
\newcommand{\mfc}{\mathfrak{c}}	\newcommand{\mfd}{\mathfrak{d}}
\newcommand{\mfe}{\mathfrak{e}}	\newcommand{\mff}{\mathfrak{f}}
\newcommand{\mfg}{\mathfrak{g}}	\newcommand{\mfh}{\mathfrak{h}}
\newcommand{\mfi}{\mathfrak{i}}	\newcommand{\mfj}{\mathfrak{j}}
\newcommand{\mfk}{\mathfrak{k}}	\newcommand{\mfl}{\mathfrak{l}}
\newcommand{\mfm}{\mathfrak{m}}	\newcommand{\mfn}{\mathfrak{n}}
\newcommand{\mfo}{\mathfrak{o}}	\newcommand{\mfp}{\mathfrak{p}}
\newcommand{\mfq}{\mathfrak{q}}	\newcommand{\mfr}{\mathfrak{r}}
\newcommand{\mfs}{\mathfrak{s}}	\newcommand{\mft}{\mathfrak{t}}
\newcommand{\mfu}{\mathfrak{u}}	\newcommand{\mfv}{\mathfrak{v}}
\newcommand{\mfw}{\mathfrak{w}}	\newcommand{\mfx}{\mathfrak{x}}
\newcommand{\mfy}{\mathfrak{y}}	\newcommand{\mfz}{\mathfrak{z}}


%---------------------------------------
% Math Roman Font
%---------------------------------------

%Captital Letters
\newcommand{\mrA}{\mathrm{A}}	\newcommand{\mrB}{\mathrm{B}}
\newcommand{\mrC}{\mathrm{C}}	\newcommand{\mrD}{\mathrm{D}}
\newcommand{\mrE}{\mathrm{E}}	\newcommand{\mrF}{\mathrm{F}}
\newcommand{\mrG}{\mathrm{G}}	\newcommand{\mrH}{\mathrm{H}}
\newcommand{\mrI}{\mathrm{I}}	\newcommand{\mrJ}{\mathrm{J}}
\newcommand{\mrK}{\mathrm{K}}	\newcommand{\mrL}{\mathrm{L}}
\newcommand{\mrM}{\mathrm{M}}	\newcommand{\mrN}{\mathrm{N}}
\newcommand{\mrO}{\mathrm{O}}	\newcommand{\mrP}{\mathrm{P}}
\newcommand{\mrQ}{\mathrm{Q}}	\newcommand{\mrR}{\mathrm{R}}
\newcommand{\mrS}{\mathrm{S}}	\newcommand{\mrT}{\mathrm{T}}
\newcommand{\mrU}{\mathrm{U}}	\newcommand{\mrV}{\mathrm{V}}
\newcommand{\mrW}{\mathrm{W}}	\newcommand{\mrX}{\mathrm{X}}
\newcommand{\mrY}{\mathrm{Y}}	\newcommand{\mrZ}{\mathrm{Z}}
%Small Letters
\newcommand{\mra}{\mathrm{a}}	\newcommand{\mrb}{\mathrm{b}}
\newcommand{\mrc}{\mathrm{c}}	\newcommand{\mrd}{\mathrm{d}}
\newcommand{\mre}{\mathrm{e}}	\newcommand{\mrf}{\mathrm{f}}
\newcommand{\mrg}{\mathrm{g}}	\newcommand{\mrh}{\mathrm{h}}
\newcommand{\mri}{\mathrm{i}}	\newcommand{\mrj}{\mathrm{j}}
\newcommand{\mrk}{\mathrm{k}}	\newcommand{\mrl}{\mathrm{l}}
\newcommand{\mrm}{\mathrm{m}}	\newcommand{\mrn}{\mathrm{n}}
\newcommand{\mro}{\mathrm{o}}	\newcommand{\mrp}{\mathrm{p}}
\newcommand{\mrq}{\mathrm{q}}	\newcommand{\mrr}{\mathrm{r}}
\newcommand{\mrs}{\mathrm{s}}	\newcommand{\mrt}{\mathrm{t}}
\newcommand{\mru}{\mathrm{u}}	\newcommand{\mrv}{\mathrm{v}}
\newcommand{\mrw}{\mathrm{w}}	\newcommand{\mrx}{\mathrm{x}}
\newcommand{\mry}{\mathrm{y}}	\newcommand{\mrz}{\mathrm{z}}

\ctexset{
    today=old
}

\title{\Huge{Linear Algebra}\\ Assignment 3}
\author{\Large{110307039\ 財管四\ 黃柏維}}
\date{\today}

\begin{document}

\maketitle
\cleardoublepage

\qs{}{(CH3.1 P116 Q18) Find the determinant of the maxtrix \( A \) using the method of expansion by cofactors.
    \begin{enumerate}[label=(\alph*)]
        \item the third row and
        \item the first column.
    \end{enumerate}, where
    \[
        A = \begin{bmatrix}
            - 3 & 4   & 2   \\
            6\, & 3\, & 1\, \\
            4\, & - 7 & - 8
        \end{bmatrix}.
    \]
}
\solve{\begin{enumerate}[label=(\alph*)]
        \item \[
                  \begin{align*}
                      \det(A) & = 4 \begin{vmatrix}
                                        4 & 2 \\
                                        3 & 1
                                    \end{vmatrix} + 7 \begin{vmatrix}
                                                          - 3 & 2 \\
                                                          6   & 1
                                                      \end{vmatrix} - 8 \begin{vmatrix}
                                                                            - 3 & 4 \\
                                                                            6   & 3
                                                                        \end{vmatrix}
                              & = 4(4 - 6) + 7(- 3 - 12) - 8(- 9 - 24)
                              & = - 8 - 105 + 264
                              & = 151.
                  \end{align*}
              \]

        \item \[
                  \begin{align*}
                      \det(A) & = - 3 \begin{vmatrix}
                                          3   & 1   \\
                                          - 7 & - 8
                                      \end{vmatrix} - 6 \begin{vmatrix}
                                                            4   & 2   \\
                                                            - 7 & - 8
                                                        \end{vmatrix} + 4 \begin{vmatrix}
                                                                              4 & 2 \\
                                                                              3 & 1
                                                                          \end{vmatrix}
                              & = - 3(- 24 + 7) - 6(- 32 + 14) + 4(4 - 6)
                              & = - 91 + 108 - 8
                              & = 9.
                  \end{align*}
              \]
    \end{enumerate}}

\qs{}{(CH3.1 P117 Q52) Find the values of \( \lambda \) for which the determinant is zero.
    \[
        \begin{vmatrix}
            \lambda & 0       & 1           \\
            0       & \lambda & 3\,         \\
            2       & 2       & \lambda - 2
        \end{vmatrix}
    \]
}
\solve{
    \[
        \begin{align*}
            \mathrm{determinant} & = \lambda \begin{vmatrix}
                                                 \lambda & 3           \\
                                                 2       & \lambda - 2
                                             \end{vmatrix} - 0 + 1 \begin{vmatrix}
                                                                       0 & \lambda \\
                                                                       2 & 2
                                                                   \end{vmatrix}
                                 & = \lambda(\lambda(\lambda - 2) - 6) - 2 \lambda
                                 & = (\lambda + 2 )\lambda (\lambda - 4)
        \end{align*}
    \]
    Therefore, for \( \lambda = - 2, 0, 4 \), the determinant is zero. }

\qs{}{(CH3.2 P124 Q30) Use elementary row or column operations to find the determinant.
    \[
        \begin{vmatrix}
            3 & 8   & - 7 \\
            0 & - 5 & 4\, \\
            6 & 1\, & - 6 \\
        \end{vmatrix}
    \]
}
\solve{
    \[
        \begin{align*}
            \begin{vmatrix}
                3 & 8   & - 7 \\
                0 & - 5 & 4\, \\
                6 & 1\, & - 6 \\
            \end{vmatrix} & = \begin{vmatrix}
                                  3 & 8    & - 7 \\
                                  0 & - 5  & 4\, \\
                                  0 & - 15 & 8
                              \end{vmatrix}
                             & = \begin{vmatrix}
                                     3 & 8   & - 7 \\
                                     0 & - 5 & 4\, \\
                                     0 & 0   & - 8
                                 \end{vmatrix}
                             & = 3 \cdot (- 5) \cdot (- 8)
                             & = 120.
        \end{align*}
    \]}

\qs{}{(CH3.3 P131 Q10) Use the fact that $ \left| cA \right| = c^n\left| A \right| $ to evaluate the determinant of the $ n \times n $ matrix $ A $.
    \[
        A = \begin{vmatrix}
            4  & 16  & 0\, \\
            12 & - 8 & 8   \\
            16 & 20  & - 4
        \end{vmatrix}
    \]
}
\solve{\[
        \begin{align*}
            \left| A \right|                                 & = 4^3 \left|
            \begin{vmatrix}
                1 & 4   & 0   \\
                3 & - 2 & 2   \\
                4 & 5   & - 1
            \end{vmatrix}
            \right|                                          & =
            64 \cdot \left( 1 \cdot \begin{vmatrix}
                                        - 2 & 2   \\
                                        5   & - 1
                                    \end{vmatrix} - 4 \cdot \begin{vmatrix}
                                                                3 & 2   \\
                                                                4 & - 1
                                                            \end{vmatrix} \right)          & =
            64 \cdot ( 1 \cdot (2 - 10) - 4 \cdot (- 3 - 8)) & =
            64 \cdot 36                                      & = 2304.
        \end{align*}
    \]
}

\qs{}{(CH3.3 P132 Q48) Find
    \begin{enumerate}[label=(\alph*)]
        \item \( \left| A^T \right| \)
        \item \( \left| A^{2} \right| \)
        \item \( \left| AA^{T} \right| \)
        \item \( \left| 2A \right| \)
        \item \( \left| A^{- 1} \right| \)
    \end{enumerate}
    for the matrix $ A $.
    \[
        A = \begin{vmatrix}
            4   & 1 & 9\, \\
            - 1 & 0 & - 2 \\
            - 3 & 3 & 0
        \end{vmatrix}
    \]
}
\solve{\begin{enumerate}[label=(\alph*)]
        \item \[
                  \begin{align*}
                      \left| A^T \right| = \left| A \right|
                       & = 1 \cdot \begin{vmatrix}
                                       1 & 9 \\
                                       3 & 0
                                   \end{vmatrix} + 0 + 2 \cdot \begin{vmatrix}
                                                                   4\, & 1 \\
                                                                   - 3 & 3
                                                               \end{vmatrix}
                       & = - 27 + 30 = 3.
                  \end{align*}
              \]
        \item \[
                  \begin{align*}
                      \left| A^2 \right| = \left| A \right|^2
                       & = 3^2 = 9.
                  \end{align*}
              \]
        \item \[
                  \begin{align*}
                      \left| AA^T \right| = \left| A \right| \cdot \left| A^T \right|
                       & = 3 \cdot 3 = 9.
                  \end{align*}
              \]
        \item \[
                  \begin{align*}
                      \left| 2A \right| = 2^3 \cdot \left| A \right|
                       & = 8 \cdot 3 = 24.
                  \end{align*}
              \]
        \item \[
                  \begin{align*}
                      \left| A^{- 1} \right| = \frac{1}{\left| A \right|}
                       & = \frac{1}{3}.
                  \end{align*}
              \]
    \end{enumerate}}

\qs{}{(CH3.4 P142 Q8) Find the adjoint of the matrix $ A $. Then use the adjoint to find the inverse of $ A $ (if possible)).
    \[
        A = \begin{bmatrix}
            1 & 1 & 1 & 0 \\
            1 & 1 & 0 & 1 \\
            1 & 0 & 1 & 1 \\
            0 & 1 & 1 & 1
        \end{bmatrix}
    \]
}
\solve{\begin{enumerate}
        \item \[
                  \begin{align*}
                      \mathrm{adj}(A) & = \begin{bmatrix}
                                              - 1 & - 1 & - 1 & 2   \\
                                              - 1 & - 1 & 2   & - 1 \\
                                              - 1 & 2   & - 1 & - 1 \\
                                              2   & - 1 & - 1 & - 1
                                          \end{bmatrix}.
                  \end{align*}
              \]
        \item By \( A^{- 1} = \frac{1}{\det(A)} \cdot \mathrm{adj}(A) \), we have \[
                  A^{-1} = \frac{1}{ - 3} \begin{bmatrix}
                      - 1 & - 1 & - 1 & 2   \\
                      - 1 & - 1 & 2   & - 1 \\
                      - 1 & 2   & - 1 & - 1 \\
                      2   & - 1 & - 1 & - 1
                  \end{bmatrix} = \begin{bmatrix}
                      \frac{1}{3}   & \frac{1}{3}   & \frac{1}{3}   & - \frac{2}{3} \\
                      \frac{1}{3}   & \frac{1}{3}   & - \frac{2}{3} & \frac{1}{3}   \\
                      \frac{1}{3}   & - \frac{2}{3} & \frac{1}{3}   & \frac{1}{3}   \\
                      - \frac{2}{3} & \frac{1}{3}   & \frac{1}{3}   & \frac{1}{3}
                  \end{bmatrix}.
              \]
    \end{enumerate}}

\qs{}{(CH3.4 P142 Q18) Use Cramer's Rule to solve (if possible) the system of equations.
    \[
        \begin{aligned}
            4x - 2y + 3z & = - 2  \\
            2x + 2y + 5z & = 16\, \\
            8x - 5y - 2z & = 4
        \end{aligned}
    \]
}
\solve{
    \begin{enumerate}
        \item Solve for $ \Delta $ \[
                  \begin{align*}
                      \Delta & = \begin{vmatrix}
                                     4 & - 2 & 3   \\
                                     2 & 2   & 5   \\
                                     8 & - 5 & - 2
                                 \end{vmatrix}
                             & = 4 \begin{vmatrix}
                                       2   & 5   \\
                                       - 5 & - 2
                                   \end{vmatrix} + 2 \begin{vmatrix}
                                                         2 & 5   \\
                                                         8 & - 2
                                                     \end{vmatrix} + 3 \begin{vmatrix}
                                                                           2 & 2   \\
                                                                           8 & - 5
                                                                       \end{vmatrix} & = 4( -4 + 25) + 2(- 4 - 40) + 3(- 10 - 16) & = - 82.
                  \end{align*}
              \]
        \item Solve for $ \Delta_x $ to find $ x $ \[
                  \begin{align*}
                      \Delta_x & = \begin{vmatrix}
                                       - 2 & - 2 & 5   \\
                                       16  & 2   & 5   \\
                                       4   & - 5 & - 2
                                   \end{vmatrix}                                         \\
                               & = - 2 \begin{vmatrix}
                                           2   & 5   \\
                                           - 5 & - 2
                                       \end{vmatrix} + 2 \begin{vmatrix}
                                                             16 & 5   \\
                                                             4  & - 2
                                                         \end{vmatrix} + 5 \begin{vmatrix}
                                                                               16 & 2   \\
                                                                               4  & - 5
                                                                           \end{vmatrix}\, \\
                               & = - 2( - 4 + 25) + 2( - 32 - 20) + 5( - 80 - 8) & = - 586
                  \end{align*}
              \]
              Therefore, $ x = \frac{\Delta_x}{\Delta} = \frac{- 586}{- 82} = \frac{293}{41}
              $.

        \item Solve for $ \Delta_y $ to find $ y $ \[
                  \begin{align*}
                      \Delta_y & = \begin{vmatrix}
                                       4 & - 2 & 3   \\
                                       2 & 16  & 5   \\
                                       8 & 4   & - 2
                                   \end{vmatrix}                                       \\
                               & = 4 \begin{vmatrix}
                                         16 & 5   \\
                                         4  & - 2
                                     \end{vmatrix} + 2 \begin{vmatrix}
                                                           2 & 5   \\
                                                           8 & - 2
                                                       \end{vmatrix} + 3 \begin{vmatrix}
                                                                             2 & 16 \\
                                                                             8 & 4
                                                                         \end{vmatrix} \\
                               & = 4( - 32 - 20) + 2( - 4 - 40) + 3(8 - 128) & = - 656
                  \end{align*}
              \]
              Therefore, $ y = \frac{\Delta_y}{\Delta} = \frac{- 656}{- 82} = 8 $.

        \item Solve for $ \Delta_z $ to find $ z $ \[
                  \begin{align*}
                      \Delta_z & = \begin{vmatrix}
                                       4 & - 2 & - 2 \\
                                       2 & 2   & 16  \\
                                       8 & - 5 & 4
                                   \end{vmatrix}                                       \\
                               & = 4 \begin{vmatrix}
                                         2   & 16 \\
                                         - 5 & 4
                                     \end{vmatrix} + 2 \begin{vmatrix}
                                                           4 & 16 \\
                                                           8 & 4
                                                       \end{vmatrix} - 2 \begin{vmatrix}
                                                                             2 & 2   \\
                                                                             8 & - 5
                                                                         \end{vmatrix} \\
                               & = 4(8 + 80) + 2(16 - 128) - 2( - 10 - 16) & = 180
                  \end{align*}
              \]
              Therefore, $ z = \frac{\Delta_z}{\Delta} = \frac{180}{- 82} = - \frac{90}{41}
              $.
    \end{enumerate}}

\qs{}{(CH3.4 P142 Q36) Determine whether the points are collinear.
    \[
        ( - 1, - 3), ( - 4, 7), (2, - 13)
    \]
}
\solve{
    \[
        \begin{align*}
            \begin{vmatrix}
                - 1 & - 3  & 1 \\
                - 4 & 7    & 1 \\
                2   & - 13 & 1
            \end{vmatrix} & = \begin{vmatrix}
                                  - 1 & - 3  & 1   \\
                                  0   & 19   & - 3 \\
                                  0   & - 19 & 3
                              \end{vmatrix} & = 0.
        \end{align*}
    \]
    Since the determinant is zero, the points are collinear. }

\qs{}{(CH3.4 P142 Q48) Determine whether the points are coplanar.
    \[
        (1, 2,3), ( - 1, 0,1), (0, - 2, - 5), (2,6,11)
    \]
}
\solve{
    \[
        \begin{align*}
            \begin{vmatrix}
                1   & 2   & 3   & 1 \\
                - 1 & 0   & 1   & 1 \\
                0\, & - 2 & - 5 & 1 \\
                2   & 6   & 11  & 1
            \end{vmatrix} & = \begin{vmatrix}
                                  1   & 2   & 3   & 1   \\
                                  0   & 2\,  & 4   & 2   \\
                                  0\, & - 2 & - 5 & 1   \\
                                  0\, & 2   & 5   & - 1
                              \end{vmatrix} & = 0.
        \end{align*}
    \]
    Since the determinant is zero, the points are coplanar. }

\qs{}{(CH4.2 P166 Q24) Determine whether the set \[
        V = \left\{ \left.\left( x, \frac{1}{2}x \right)\, \right| x \text{ is a real number}\right\}
    \] together with the standard operations, is a vector space. If it is not,
    identify at least one of the ten vector space axioms that fails. 
} 
\solve{
    Let's exaimine the ten vector space axioms.
    \begin{enumerate}
        \item \( \forall\,  \bmu, \bmv \in V,\, \bmu + \bmv \in V \) \\
              Let \( \bmu = \left( x, \frac{1}{2}x \right) \) and \( \bmv = \left( y, \frac{1}{2}y \right) \in V. \) \\
              Then \( \bmu + \bmv = \left( x + y, \frac{1}{2}x + \frac{1}{2}y \right) \in V. \)
        \item \( \forall\,  \bmu, \bmv \in V,\, \bmu + \bmv = \bmv + \bmu \) \\
              Let \( \bmu = \left( x, \frac{1}{2}x \right) \) and \( \bmv = \left( y, \frac{1}{2}y \right) \in V. \) \\
              Then \( \bmu + \bmv = \left( x + y, \frac{1}{2}x + \frac{1}{2}y \right) = \left( y + x, \frac{1}{2}y + \frac{1}{2}x \right) = \bmv + \bmu. \)
        \item \( \forall\,  \bmu, \bmv, \bmw \in V,\, \bmu + (\bmv + \bmw) = (\bmu + \bmv) + \bmw \) \\
              Let \( \bmu = \left( x, \frac{1}{2}x \right), \bmv = \left( y, \frac{1}{2}y \right), \bmw = \left( z, \frac{1}{2}z \right) \in V. \) \\
              Then \( \bmu + (\bmv + \bmw) = \left( x, \frac{1}{2}x \right) + \left( y + z, \frac{1}{2}y + \frac{1}{2}z \right) = \left( x + y + z, \frac{1}{2}x + \frac{1}{2}y + \frac{1}{2}z \right) \) \\
              and \( (\bmu + \bmv) + \bmw = \left( x + y, \frac{1}{2}x + \frac{1}{2}y \right) + \left( z, \frac{1}{2}z \right) = \left( x + y + z, \frac{1}{2}x + \frac{1}{2}y + \frac{1}{2}z \right). \)
        \item \( \exists\,\boldsymbol{0}  \in V,\, \text{s.t. } \forall\,  \bmu \in V,\, \bmu + \boldsymbol{0} = \bmu \) \\
                Let \( \bmu = \left( x, \frac{1}{2}x \right) \text{ and } \boldsymbol{0} = \left( 0, 0 \right) \in V \) \\
                Then \( \bmu + \boldsymbol{0} = \left( x, \frac{1}{2}x \right) + \left( 0, 0 \right) = \left( x, \frac{1}{2}x \right) = \bmu. \)
        \item \( \forall\,  \bmu \in V,\, \exists\, - \bmu \in V,\, \text{s.t. } \bmu + ( - \bmu) = \boldsymbol{0} \) \\
                Let \( \bmu = \left( x, \frac{1}{2}x \right) \in V \) \\
                Then \( - \bmu = \left( - x, - \frac{1}{2}x \right) \in V \) \\
                And \( \bmu + ( - \bmu) = \left( x, \frac{1}{2}x \right) + \left( - x, - \frac{1}{2}x \right) = \left( 0, 0 \right) = \boldsymbol{0}. \)
        \item \( \forall\, \bmu \in V,\, \forall\, c \in \bbR ,\, c \cdot \bmu \in V \) \\
                Let \( \bmu = \left( x, \frac{1}{2}x \right) \in V \) and \( c \in \bbR \) \\
                Then \( c \cdot \bmu = c \cdot \left( x, \frac{1}{2}x \right) = \left( c \cdot x, c \cdot \frac{1}{2}x \right) \in V. \)
        \item \( \forall\, \bmu, \bmv \in V,\, c \in \bbR ,\, c \cdot (\bmu + \bmv) = c \cdot \bmu + c \cdot \bmv \) \\
                Let \( \bmu = \left( x, \frac{1}{2}x \right), \bmv = \left( y, \frac{1}{2}y \right) \in V \) and \( c \in \bbR \) \\
                Then \( c \cdot (\bmu + \bmv) = c \cdot \left( x + y, \frac{1}{2}x + \frac{1}{2}y \right) = \left( c \cdot (x + y), c \cdot \frac{1}{2}x + c \cdot \frac{1}{2}y \right) \) \\
                and \( c \cdot \bmu + c \cdot \bmv = \left( c \cdot x, c \cdot \frac{1}{2}x \right) + \left( c \cdot y, c \cdot \frac{1}{2}y \right) = \left( c \cdot (x + y), c \cdot \frac{1}{2}x + c \cdot \frac{1}{2}y \right). \)
        \item \( \forall\, \bmu \in V,\, \forall\, c, d \in \bbR ,\, (c + d) \cdot \bmu = c \cdot \bmu + d \cdot \bmu \) \\
                Let \( \bmu = \left( x, \frac{1}{2}x \right) \in V \) and \( c, d \in \bbR \) \\
                Then \( (c + d) \cdot \bmu = (c + d) \cdot \left( x, \frac{1}{2}x \right) = \left( (c + d) \cdot x, (c + d) \cdot \frac{1}{2}x \right) \) \\
                and \( c \cdot \bmu + d \cdot \bmu = \left( c \cdot x, c \cdot \frac{1}{2}x \right) + \left( d \cdot x, d \cdot \frac{1}{2}x \right) = \left( (c + d) \cdot x, (c + d) \cdot \frac{1}{2}x \right). \)
        \item \( \forall\, \bmu \in V,\, \forall\, c, d \in  \bbR,\, c(d\bmu) = (cd) \bmu \) \\
                Let \( \bmu = \left( x, \frac{1}{2}x \right) \in V \) and \( c, d \in \bbR \) \\
                Then \( c(d\bmu) = c(d \cdot \left( x, \frac{1}{2}x \right)) = c \cdot \left( d \cdot x, d \cdot \frac{1}{2}x \right) = \left( c \cdot d \cdot x, c \cdot d \cdot \frac{1}{2}x \right) \) \\
                and \( (cd) \bmu = (cd) \cdot \left( x, \frac{1}{2}x \right) = \left( cd \cdot x, cd \cdot \frac{1}{2}x \right). \)
        \item \( \forall\, \bmu \in V,\, 1 \cdot \bmu = \bmu \) \\
                Let \( \bmu = \left( x, \frac{1}{2}x \right) \in V \) \\
                Then \( 1 \cdot \bmu = 1 \cdot \left( x, \frac{1}{2}x \right) = \left( 1 \cdot x, 1 \cdot \frac{1}{2}x \right) = \left( x, \frac{1}{2}x \right) = \bmu. \)
    \end{enumerate}
    Since all the ten vector space axioms hold, the set \( V \) together with the standard operations is a vector space. 
}

\qs{}{(CH4.2 P166 Q26) Determine whether the set \[
    V = \left\{ \left.\begin{bmatrix}
        a & b \\
        c & 0 
    \end{bmatrix}\, \right| a, b, c \in F \right\}
\] together with the standard operations, is a vector space. If it is not,
identify at least one of the ten vector space axioms that fails. 
}
\solve{
    Let's exaimine the ten vector space axioms.
    For the ease of discussion, assume \( A = \begin{bmatrix}
        a & b \\
        c & 0
    \end{bmatrix} \) and \( B = \begin{bmatrix}
        x & y \\
        z & 0
    \end{bmatrix} \) and \( C = \begin{bmatrix}
        m & n \\
        p & 0
    \end{bmatrix} \) are arbitrary elements in \( V \).
    And \( c, d\) are arbitrary elements in \( \bbR \).
    \begin{enumerate}
        \item \( A + B \in V \) \\
              \( A + B = \begin{bmatrix}
                  a + x & b + y \\
                  c + z & 0
              \end{bmatrix} \in V. \)
        \item \( A + B = B + A \) \\
                
                \( A + B = \begin{bmatrix}
                    a + x & b + y \\
                    c + z & 0
                \end{bmatrix} = \begin{bmatrix}
                    x + a & y + b \\
                    z + c & 0
                \end{bmatrix} = B + A. \)
        \item \( A + (B + C) = (A + B) + C \) \\
                \( A + (B + C) = \begin{bmatrix}
                    a & b \\
                    c & 0
                \end{bmatrix} + \begin{bmatrix}
                    x + m & y + n \\
                    z + p & 0
                \end{bmatrix} = \begin{bmatrix}
                    a + x + m & b + y + n \\
                    c + z + p & 0
                \end{bmatrix} \) \\
                and \( (A + B) + C = \begin{bmatrix}
                    a + x & b + y \\
                    c + z & 0
                \end{bmatrix} + \begin{bmatrix}
                    m & n \\
                    p & 0
                \end{bmatrix} = \begin{bmatrix}
                    a + x + m & b + y + n \\
                    c + z + p & 0
                \end{bmatrix}. \) 
        \item \( \exists\, \boldsymbol{0} \in V,\, \text{s.t. } \forall\, A \in V,\, A + \boldsymbol{0} = A \) \\
                Let \( \boldsymbol{0} = \begin{bmatrix}
                    0 & 0 \\
                    0 & 0
                \end{bmatrix} \in V \) \\
                Then \( A + \boldsymbol{0} = \begin{bmatrix}
                    a & b \\
                    c & 0
                \end{bmatrix} + \begin{bmatrix}
                    0 & 0 \\
                    0 & 0
                \end{bmatrix} = \begin{bmatrix}
                    a & b \\
                    c & 0
                \end{bmatrix} = A. \)
        \item \( \forall\, A \in V,\, \exists\, - A \in V,\, \text{s.t. } A + ( - A) = \boldsymbol{0} \) \\
                Let \( - A = \begin{bmatrix}
                    - a & - b \\
                    - c & 0
                \end{bmatrix} \in V \) \\
                Then \( A + ( - A) = \begin{bmatrix}
                    a & b \\
                    c & 0
                \end{bmatrix} + \begin{bmatrix}
                    - a & - b \\
                    - c & 0
                \end{bmatrix} = \begin{bmatrix}
                    0 & 0 \\
                    0 & 0
                \end{bmatrix} = \boldsymbol{0}. \)
        \item \( c \cdot A \in V \) \\
                \( c \cdot A = \begin{bmatrix}
                    c \cdot a & c \cdot b \\
                    c \cdot c & 0
                \end{bmatrix} \in V. \)
        \item \( c \cdot (A + B) = c \cdot A + c \cdot B \) \\
                \( c \cdot (A + B) = c \cdot \begin{bmatrix}
                    a + x & b + y \\
                    c + z & 0
                \end{bmatrix} = \begin{bmatrix}
                    c \cdot (a + x) & c \cdot (b + y) \\
                    c \cdot (c + z) & 0
                \end{bmatrix} \) \\
                and \( c \cdot A + c \cdot B = \begin{bmatrix}
                    c \cdot a & c \cdot b \\
                    c \cdot c & 0
                \end{bmatrix} + \begin{bmatrix}
                    c \cdot x & c \cdot y \\
                    c \cdot z & 0
                \end{bmatrix} = \begin{bmatrix}
                    c \cdot (a + x) & c \cdot (b + y) \\
                    c \cdot (c + z) & 0
                \end{bmatrix}. \)
        \item \( (c + d) \cdot A = c \cdot A + d \cdot A \) \\
                \( (c + d) \cdot A = (c + d) \cdot \begin{bmatrix}
                    a & b \\
                    c & 0
                \end{bmatrix} = \begin{bmatrix}
                    (c + d) \cdot a & (c + d) \cdot b \\
                    (c + d) \cdot c & 0
                \end{bmatrix} \) \\
                and \( c \cdot A + d \cdot A = \begin{bmatrix}
                    c \cdot a & c \cdot b \\
                    c \cdot c & 0
                \end{bmatrix} + \begin{bmatrix}
                    d \cdot a & d \cdot b \\
                    d \cdot c & 0
                \end{bmatrix} = \begin{bmatrix}
                    (c + d) \cdot a & (c + d) \cdot b \\
                    (c + d) \cdot c & 0
                \end{bmatrix}. \)
        \item \( c \cdot (d \cdot A) = (cd) \cdot A \) \\
                \( c \cdot (d \cdot A) = c \cdot \begin{bmatrix}
                    d \cdot a & d \cdot b \\
                    d \cdot c & 0
                \end{bmatrix} = \begin{bmatrix}
                    c \cdot (d \cdot a) & c \cdot (d \cdot b) \\
                    c \cdot (d \cdot c) & 0
                \end{bmatrix} \) \\
                and \( (cd) \cdot A = (cd) \cdot \begin{bmatrix}
                    a & b \\
                    c & 0
                \end{bmatrix} = \begin{bmatrix}
                    (cd) \cdot a & (cd) \cdot b \\
                    (cd) \cdot c & 0
                \end{bmatrix}. \)
        \item \( 1 \cdot A = A \) \\
                \( 1 \cdot A = 1 \cdot \begin{bmatrix}
                    a & b \\
                    c & 0
                \end{bmatrix} = \begin{bmatrix}
                    1 \cdot a & 1 \cdot b \\
                    1 \cdot c & 0
                \end{bmatrix} = \begin{bmatrix}
                    a & b \\
                    c & 0
                \end{bmatrix} = A. \)   
    \end{enumerate}
    Since all the ten vector space axioms hold, the set \( V \) together with the standard operations is a vector space.
}

\qs{}{(CH4.2 P166 Q30) Determine whether the set \[
    V = \left\{ \left.\begin{bmatrix}
        0 & a & b & c \\
        a & 0 & b & c \\
        a & b & 0 & c \\
        a & b & c & 0
    \end{bmatrix}\, \right| a, b, c \in F \right\}
\] together with the standard operations, is a vector space. If it is not,
identify at least one of the ten vector space axioms that fails. 
}
\solve{
    Let's exaimine the ten vector space axioms.
    For the ease of discussion, assume \( A = \begin{bmatrix}
        0 & a & b & c \\
        a & 0 & b & c \\
        a & b & 0 & c \\
        a & b & c & 0
    \end{bmatrix} \) and \( B = \begin{bmatrix}
        0 & x & y & z \\
        x & 0 & y & z \\
        x & y & 0 & z \\
        x & y & z & 0
    \end{bmatrix} \) and \( C = \begin{bmatrix}
        0 & m & n & p \\
        m & 0 & n & p \\
        m & n & 0 & p \\
        m & n & p & 0
    \end{bmatrix} \) are arbitrary elements in \( V \).
    And \( c, d\) are arbitrary elements in \( \bbR \).
    \begin{enumerate}
        \item \( A + B \in V \) \\
              \( A + B = \begin{bmatrix}
                  0 & a + x & b + y & c + z \\
                  a + x & 0 & b + y & c + z \\
                  a + x & b + y & 0 & c + z \\
                  a + x & b + y & c + z & 0
              \end{bmatrix} \in V. \)

        \item \( A + B = B + A \) \\
                \( 
                    A + B & = \begin{bmatrix}
                        0 & a & b & c \\
                        a & 0 & b & c \\
                        a & b & 0 & c \\
                        a & b & c & 0
                    \end{bmatrix} + \begin{bmatrix}
                        0 & x & y & z \\
                        x & 0 & y & z \\
                        x & y & 0 & z \\
                        x & y & z & 0
                    \end{bmatrix} \\
                    = \begin{bmatrix}
                        0 & a + x & b + y & c + z \\
                        a + x & 0 & b + y & c + z \\
                        a + x & b + y & 0 & c + z \\
                        a + x & b + y & c + z & 0
                    \end{bmatrix} = \begin{bmatrix}
                        0 & x + a & y + b & z + c \\
                        x + a & 0 & y + b & z + c \\
                        x + a & y + b & 0 & z + c \\
                        x + a & y + b & z + c & 0
                    \end{bmatrix} = B + A.
                
                \)

        \item \( A + (B + C) = (A + B) + C \) \\
                \( \begin{aligned} A + (B + C) &= \begin{bmatrix}
                    0 & a & b & c \\
                    a & 0 & b & c \\
                    a & b & 0 & c \\
                    a & b & c & 0
                \end{bmatrix} + \begin{bmatrix}
                    0 & x + m & y + n & z + p \\
                    x + m & 0 & y + n & z + p \\
                    x + m & y + n & 0 & z + p \\
                    x + m & y + n & z + p & 0
                \end{bmatrix} \\
                &= \begin{bmatrix}
                    0 & a + x + m & b + y + n & c + z + p \\
                    a + x + m & 0 & b + y + n & c + z + p \\
                    a + x + m & b + y + n & 0 & c + z + p \\
                    a + x + m & b + y + n & c + z + p & 0
                \end{bmatrix} 
            \end{aligned}\) \\
                \( \begin{aligned} \text{and } (A + B) + C &= \begin{bmatrix}
                    0 & a + x & b + y & c + z \\
                    a + x & 0 & b + y & c + z \\
                    a + x & b + y & 0 & c + z \\
                    a + x & b + y & c + z & 0
                \end{bmatrix} + \begin{bmatrix}
                    0 & m & n & p \\
                    m & 0 & n & p \\
                    m & n & 0 & p \\
                    m & n & p & 0
                \end{bmatrix} \\
                &= \begin{bmatrix}
                    0 & a + x + m & b + y + n & c + z + p \\
                    a + x + m & 0 & b + y + n & c + z + p \\
                    a + x + m & b + y + n & 0 & c + z + p \\
                    a + x + m & b + y + n & c + z + p & 0
                \end{bmatrix}. 
            \end{aligned} \)
        \item \( \exists\, \boldsymbol{0} \in V,\, \text{s.t. } \forall\, A \in V,\, A + \boldsymbol{0} = A \) \\
                Let \( \boldsymbol{0} = \begin{bmatrix}
                    0 & 0 & 0 & 0 \\
                    0 & 0 & 0 & 0 \\
                    0 & 0 & 0 & 0 \\
                    0 & 0 & 0 & 0
                \end{bmatrix} \in V \) \\
                Then \( A + \boldsymbol{0} = \begin{bmatrix}
                    0 & a & b & c \\
                    a & 0 & b & c \\
                    a & b & 0 & c \\
                    a & b & c & 0
                \end{bmatrix} + \begin{bmatrix}
                    0 & 0 & 0 & 0 \\
                    0 & 0 & 0 & 0 \\
                    0 & 0 & 0 & 0 \\
                    0 & 0 & 0 & 0
                \end{bmatrix} = \begin{bmatrix}
                    0 & a & b & c \\
                    a & 0 & b & c \\
                    a & b & 0 & c \\
                    a & b & c & 0
                \end{bmatrix} = A. \)
        \item \( \forall\, A \in V,\, \exists\, - A \in V,\, \text{s.t. } A + ( - A) = \boldsymbol{0} \) \\
                Let \( - A = \begin{bmatrix}
                    0 & - a & - b & - c \\
                    - a & 0 & - b & - c \\
                    - a & - b & 0 & - c \\
                    - a & - b & - c & 0
                \end{bmatrix} \in V \) \\
                Then \( A + ( - A) = \begin{bmatrix}
                    0 & a & b & c \\
                    a & 0 & b & c \\
                    a & b & 0 & c \\
                    a & b & c & 0
                \end{bmatrix} + \begin{bmatrix}
                    0 & - a & - b & - c \\
                    - a & 0 & - b & - c \\
                    - a & - b & 0 & - c \\
                    - a & - b & - c & 0
                \end{bmatrix} = \begin{bmatrix}
                    0 & 0 & 0 & 0 \\
                    0 & 0 & 0 & 0 \\
                    0 & 0 & 0 & 0 \\
                    0 & 0 & 0 & 0
                \end{bmatrix} = \boldsymbol{0}. \)
        \item \( c \cdot A \in V \) \\
                \( c \cdot A = \begin{bmatrix}
                    0 & c \cdot a & c \cdot b & c \cdot c \\
                    c \cdot a & 0 & c \cdot b & c \cdot c \\
                    c \cdot a & c \cdot b & 0 & c \cdot c \\
                    c \cdot a & c \cdot b & c \cdot c & 0
                \end{bmatrix} \in V. \)
        \item \( c \cdot (A + B) = c \cdot A + c \cdot B \) \\
                \( c \cdot (A + B) = c \cdot \begin{bmatrix}
                    0 & a + x & b + y & c + z \\
                    a + x & 0 & b + y & c + z \\
                    a + x & b + y & 0 & c + z \\
                    a + x & b + y & c + z & 0
                \end{bmatrix} = \begin{bmatrix}
                    0 & c \cdot (a + x) & c \cdot (b + y) & c \cdot (c + z) \\
                    c \cdot (a + x) & 0 & c \cdot (b + y) & c \cdot (c + z) \\
                    c \cdot (a + x) & c \cdot (b + y) & 0 & c \cdot (c + z) \\
                    c \cdot (a + x) & c \cdot (b + y) & c \cdot (c + z) & 0
                \end{bmatrix} \) \\
                \( \begin{aligned} \text{and }c \cdot A + c \cdot B &= \begin{bmatrix}
                    0 & c \cdot a & c \cdot b & c \cdot c \\
                    c \cdot a & 0 & c \cdot b & c \cdot c \\
                    c \cdot a & c \cdot b & 0 & c \cdot c \\
                    c \cdot a & c \cdot b & c \cdot c & 0
                \end{bmatrix} + \begin{bmatrix}
                    0 & c \cdot x & c \cdot y & c \cdot z \\
                    c \cdot x & 0 & c \cdot y & c \cdot z \\
                    c \cdot x & c \cdot y & 0 & c \cdot z \\
                    c \cdot x & c \cdot y & c \cdot z & 0
                \end{bmatrix} \\
                &= \begin{bmatrix}
                    0 & c \cdot (a + x) & c \cdot (b + y) & c \cdot (c + z) \\
                    c \cdot (a + x) & 0 & c \cdot (b + y) & c \cdot (c + z) \\
                    c \cdot (a + x) & c \cdot (b + y) & 0 & c \cdot (c + z) \\
                    c \cdot (a + x) & c \cdot (b + y) & c \cdot (c + z) & 0
                \end{bmatrix}. 
            \end{aligned} \)
        \item \( (c + d) \cdot A = c \cdot A + d \cdot A \) \\
                \( (c + d) \cdot A = (c + d) \cdot \begin{bmatrix}
                    0 & a & b & c \\
                    a & 0 & b & c \\
                    a & b & 0 & c \\
                    a & b & c & 0
                \end{bmatrix} = \begin{bmatrix}
                    0 & (c + d) \cdot a & (c + d) \cdot b & (c + d) \cdot c \\
                    (c + d) \cdot a & 0 & (c + d) \cdot b & (c + d) \cdot c \\
                    (c + d) \cdot a & (c + d) \cdot b & 0 & (c + d) \cdot c \\
                    (c + d) \cdot a & (c + d) \cdot b & (c + d) \cdot c & 0
                \end{bmatrix} \) \\
                \( \begin{aligned} \text{and } 
                c \cdot A + d \cdot A &= \begin{bmatrix}
                    0 & c \cdot a & c \cdot b & c \cdot c \\
                    c \cdot a & 0 & c \cdot b & c \cdot c \\
                    c \cdot a & c \cdot b & 0 & c \cdot c \\
                    c \cdot a & c \cdot b & c \cdot c & 0
                \end{bmatrix} + \begin{bmatrix}
                    0 & d \cdot a & d \cdot b & d \cdot c \\
                    d \cdot a & 0 & d \cdot b & d \cdot c \\
                    d \cdot a & d \cdot b & 0 & d \cdot c \\
                    d \cdot a & d \cdot b & d \cdot c & 0
                \end{bmatrix} \\
                &= \begin{bmatrix}
                    0 & (c + d) \cdot a & (c + d) \cdot b & (c + d) \cdot c \\
                    (c + d) \cdot a & 0 & (c + d) \cdot b & (c + d) \cdot c \\
                    (c + d) \cdot a & (c + d) \cdot b & 0 & (c + d) \cdot c \\
                    (c + d) \cdot a & (c + d) \cdot b & (c + d) \cdot c & 0
                \end{bmatrix}. 
            \end{aligned} \)
        \item \( c \cdot (d \cdot A) = (cd) \cdot A \) \\
                \( c \cdot (d \cdot A) = c \cdot \begin{bmatrix}
                    0 & d \cdot a & d \cdot b & d \cdot c \\
                    d \cdot a & 0 & d \cdot b & d \cdot c \\
                    d \cdot a & d \cdot b & 0 & d \cdot c \\
                    d \cdot a & d \cdot b & d \cdot c & 0
                \end{bmatrix} = \begin{bmatrix}
                    0 & c \cdot (d \cdot a) & c \cdot (d \cdot b) & c \cdot (d \cdot c) \\
                    c \cdot (d \cdot a) & 0 & c \cdot (d \cdot b) & c \cdot (d \cdot c) \\
                    c \cdot (d \cdot a) & c \cdot (d \cdot b) & 0 & c \cdot (d \cdot c) \\
                    c \cdot (d \cdot a) & c \cdot (d \cdot b) & c \cdot (d \cdot c) & 0
                \end{bmatrix} \) \\
                and \( (cd) \cdot A = (cd) \cdot \begin{bmatrix}
                    0 & a & b & c \\
                    a & 0 & b & c \\
                    a & b & 0 & c \\
                    a & b & c & 0
                \end{bmatrix} = \begin{bmatrix}
                    0 & (cd) \cdot a & (cd) \cdot b & (cd) \cdot c \\
                    (cd) \cdot a & 0 & (cd) \cdot b & (cd) \cdot c \\
                    (cd) \cdot a & (cd) \cdot b & 0 & (cd) \cdot c \\
                    (cd) \cdot a & (cd) \cdot b & (cd) \cdot c & 0
                \end{bmatrix}. \)
        \item \( 1 \cdot A = A \) \\
                \( 1 \cdot A = 1 \cdot \begin{bmatrix}
                    0 & a & b & c \\
                    a & 0 & b & c \\
                    a & b & 0 & c \\
                    a & b & c & 0
                \end{bmatrix} = \begin{bmatrix}
                    0 & 1 \cdot a & 1 \cdot b & 1 \cdot c \\
                    1 \cdot a & 0 & 1 \cdot b & 1 \cdot c \\
                    1 \cdot a & 1 \cdot b & 0 & 1 \cdot c \\
                    1 \cdot a & 1 \cdot b & 1 \cdot c & 0
                \end{bmatrix} = \begin{bmatrix}
                    0 & a & b & c \\
                    a & 0 & b & c \\
                    a & b & 0 & c \\
                    a & b & c & 0
                \end{bmatrix} = A. \) 
    \end{enumerate} 
    Since all the ten vector space axioms hold, the set \( V \) together with the standard operations is a vector space.
}

\qs{}{(CH4.3 P173 Q2) Verify that $ W $ is a subspace of $ V $, assume that $ V $ has the standard operations. \[
    W = \left\{ (x, y, 4x - 5y) \mid x, y \in \bbR \right\} \subseteq \bbR^3 = V.
\]
}
\solve{
    Let \( \bmu = (x_1, y_1, 4x_1 - 5y_1) \) and \( \bmv = (x_2, y_2, 4x_2 - 5y_2) \) be arbitrary elements in \( W \) and \( c, d \in \bbR \). \\
    \[ \begin{align*} c \cdot  \bmu + d \cdot \bmv &= c \cdot (x_1, y_1, 4x_1 - 5y_1) + d \cdot (x_2, y_2, 4x_2 - 5y_2) \\
        &= (c \cdot x_1 + d \cdot x_2, c \cdot y_1 + d \cdot y_2, c \cdot (4x_1 - 5y_1) + d \cdot (4x_2 - 5y_2)) \in V
    \end{align*}
    \] \\
}

\qs{}{(CH4.3 P173 Q4) Verify that $ W $ is a subspace of $ V $, assume that $ V $ has the standard operations. \[
    W = \left\{ \left. \begin{bmatrix}
        a & b \\
        a - 2b & 0 \\
        0 & c
    \end{bmatrix}\, \right| a, b, c \in \bbR \right\} \subseteq M_{3 \cdot 2}( \bbR ) = V.
\]
}
\solve{
    Let \( A = \begin{bmatrix}
        a_1 & b_1 \\
        a_1 - 2b_1 & 0 \\
        0 & c_1
    \end{bmatrix} \) and \( B = \begin{bmatrix}
        a_2 & b_2 \\
        a_2 - 2b_2 & 0 \\
        0 & c_2
    \end{bmatrix} \) be arbitrary elements in \( W \) and \( c, d \in \bbR \). \\
    \[ \begin{align*} c \cdot A + d \cdot B &= c \cdot \begin{bmatrix}
        a_1 & b_1 \\
        a_1 - 2b_1 & 0 \\
        0 & c_1
    \end{bmatrix} + d \cdot \begin{bmatrix}
        a_2 & b_2 \\
        a_2 - 2b_2 & 0 \\
        0 & c_2
    \end{bmatrix} \\
    &= \begin{bmatrix}
        c \cdot a_1 + d \cdot a_2 & c \cdot b_1 + d \cdot b_2 \\
        c \cdot (a_1 - 2b_1) + d \cdot (a_2 - 2b_2) & 0 \\
        0 & c \cdot c_1 + d \cdot c_2
    \end{bmatrix} \in V
    \end{align*} \]
}

\qs{}{(CH4.3 P173 Q32) Determine whether the subset \( W \) of \( M_{n \times n}(F) \) is a subspace of \( M_{n \times n}(F) \) with the standard operations.
    Justify your answer.\[
    W \text{ is the set of all } A \in M_{n \times n}(F) \text{ s.t. } AB = BA, \text{ for a given } B \in M_{n \times n}(F).
\]
}
\solve{ \( \forall\, A_1, A_2 \in  W,\, \forall\,  c, d \in F\) \\
    \begin{enumerate}
        \item \[ \begin{align*}
            & A_1 \in W \text{ and } A_2 \in W \\
            \implies & A_1 B = BA_1 \text{ and } A_2 B = BA_2 \\
            \implies & (A_1 + A_2)B = A_1B + A_2B = BA_1 + BA_2 = B(A_1 + A_2) \\
            \implies & A_1 + A_2 \in W
        \end{align*}
        \]

        \item \[
            & (c \cdot A_1)B = c \cdot (A_1B) = c \cdot (BA_1) = B(c \cdot A_1) \\
            \implies & c \cdot A_1 \in W
        \]
    \end{enumerate}
    Therefore, \( W \) is a subspace of \( M_{n \times n}(F) \) with the standard operations.
}

\qs{}{(CH4.3 P173 Q38) Determine whether the set \( W \)  is a subspace of \( \RR[3] \) with the standard operations.
    Justify your answer.\[
    W = \left\{ (x_1, x_2, 4) \mid x_1, x_2 \in \RR \right\}
\]
}
\solve{ \( W \) is not a subspace of \( \RR[3] \) with the standard operations. \\
    Let \( \bmu = (1, 2, 4) \in W \) and \( c = -1 \in \RR \). \\
    Then \( c \cdot \bmu = -1 \cdot (1, 2, 4) = (-1, -2, -4) \notin W \). \\
    Therefore, \( W \) is not a subspace of \( \RR[3] \) with the standard operations.
}

\end{document}