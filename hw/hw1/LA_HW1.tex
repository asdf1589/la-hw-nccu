\documentclass{report}

\usepackage{ctex}

\input{../module/preamble}
\input{../module/macros}
\input{../module/letterfonts}

\ctexset{
    today=old
}

\title{\Huge{Linear Algebra}\\ Assignment 1}
\author{\Large{110307039\ 財管四\ 黃柏維}}
\date{\today}

\begin{document}

\maketitle
\cleardoublepage

\qs{}{Find the inverse of the following matrices:
    \[
        A = \begin{bmatrix}
            -\frac{5}{6} & \frac{1}{3}  & \frac{11}{6} \\
            0            & \frac{2}{3}  & 2            \\
            1            & -\frac{1}{2} & -\frac{5}{2}
        \end{bmatrix},\quad
        B = \begin{bmatrix}
            3  & 2 & 5 \\
            2  & 2 & 4 \\
            -4 & 4 & 0
        \end{bmatrix}
    \]
}

\solve{
    \begin{enumerate}
        \item For matrix $A$, let
              \[
                  A  = \begin{bmatrix}
                      \bmA_1 \\ \bmA_2 \\ \bmA_3
                  \end{bmatrix}
              \]
              where $\bmA_1,\ \bmA_2,\ \bmA_3$ are the rows of matrix $A$. Then we have
              \[
                  -12\bmA_1 - \frac{3}{2}\bmA_2 - 10\bmA_3 = \mathbf{0},
              \]
              which implies that the rows of $A$ are linearly dependent, therefore the
              inverse of matrix $A$ does not exist.

        \item For matrix $B$, we have
              \[
                  \det(B) = \begin{vmatrix}
                      3  & 2 & 5 \\
                      2  & 2 & 4 \\
                      -4 & 4 & 0
                  \end{vmatrix}
                  = \begin{vmatrix}
                      3 & 2 & 5 \\
                      2 & 2 & 4 \\
                      0 & 8 & 8
                  \end{vmatrix}
                  = 8\cdot\begin{vmatrix}
                      3 & 0 & 3 \\
                      2 & 0 & 2 \\
                      0 & 1 & 1
                  \end{vmatrix}
                  = 0.
              \]
              Since the determinant of $B$ is zero, the inverse of matrix $B$ does not
              exists.
    \end{enumerate}
}

\qs{}{ Find $(AB)^{-1},(A^{T})^{-1}$ and $(2A)^{-1}$ where
    \[
        A = \begin{bmatrix}
            1 & -4 & 2 \\
            0 & 1  & 3 \\
            4 & 2  & 1
        \end{bmatrix},\quad
        B = \begin{bmatrix}
            1 & 0 & 0 \\
            1 & 1 & 0 \\
            0 & 1 & 1
        \end{bmatrix}
    \]
}

\solve{
    The inverse of matrix $A$ and $B$ are
    \[
        A^{-1} = \frac{1}{61}\begin{bmatrix}
            5   & -8 & -14 \\
            -12 & 7  & 3   \\
            4   & 18 & -1
        \end{bmatrix},\quad
        B^{-1} = \begin{bmatrix}
            1  & 0  & 0 \\
            -1 & 1  & 0 \\
            1  & -1 & 1
        \end{bmatrix}.
    \]
    Therefore, we have
    \begin{enumerate}
        \item $(AB)^{-1} = B^{-1}A^{-1} = \frac{1}{61}\begin{bmatrix}
                      5   & -8 & -14 \\
                      -17 & 15 & -11 \\
                      21  & 3  & 10
                  \end{bmatrix}$.
        \item $(A^{T})^{-1} = (A^{-1})^{T} = \frac{1}{61}\begin{bmatrix}
                      5   & -12 & 4  \\
                      -8  & 7   & 18 \\
                      -14 & 3   & -1
                  \end{bmatrix}$.
        \item $(2A)^{-1} = 2^{-1}A^{-1} = \frac{1}{122}\begin{bmatrix}
                      5   & -8 & -14 \\
                      -12 & 7  & 3   \\
                      4   & 18 & -1
                  \end{bmatrix}$.
    \end{enumerate}
}

\qs{}{Find a sequence of elementary matrices that can be used to write the matrix $A$ in row-echelon form, where
    \[
        A = \begin{bmatrix}
            1 & 3  & 0  \\
            2 & 5  & -1 \\
            3 & -2 & -4
        \end{bmatrix}
    \]
}
\solve{
    We firstly do the following row operations:
    \[
        1.\quad \begin{align*}
             & R_2 \leftarrow R_2 - 2R_1 \\
             & R_3 \leftarrow R_3 - 3R_1
        \end{align*}
        \implies A \stackrel{\mathrm{r}}{\sim}
        \begin{bmatrix}
            1 & 3   & 0  \\
            0 & -1  & -1 \\
            0 & -11 & -4
        \end{bmatrix}\\
    \]
    and
    \[
        2. \quad & R_3 \leftarrow R_3 - 11R_2
        \implies A \stackrel{\mathrm{r}}{\sim}
        \begin{bmatrix}
            1 & 3  & 0  \\
            0 & -1 & -1 \\
            0 & 0  & 7
        \end{bmatrix}
        \triangleq U.
    \]
    Therefore, the matrix $A$ can be written in row-echelon form by
    \[
        E_2E_1A = U,
    \]
    where
    \[
        E_1 = \begin{bmatrix}
            1  & 0 & 0 \\
            -2 & 1 & 0 \\
            -3 & 0 & 1
        \end{bmatrix},\quad
        E_2 = \begin{bmatrix}
            1 & 0  & 0 \\
            0 & 1  & 0 \\
            0 & 11 & 1
        \end{bmatrix}.
    \]
}

\qs{}{Find and LU-factorization of the matrix $A$, where
    \[
        A = \begin{bmatrix}
            2  & 0  & 0 \\
            0  & -3 & 1 \\
            10 & 12 & 3
        \end{bmatrix}
    \]
}
\solve{
    We can factorize the matrix $A$ by
    \[
        A = \begin{bmatrix}
            2  & 0  & 0 \\
            0  & -3 & 1 \\
            10 & 12 & 3
        \end{bmatrix}
        = \begin{bmatrix}
            1 & 0  & 0 \\
            0 & 1  & 0 \\
            5 & -4 & 1
        \end{bmatrix}
        \begin{bmatrix}
            2 & 0  & 0 \\
            0 & -3 & 1 \\
            0 & 0  & 7
        \end{bmatrix}
        = LU.
    \]
}

\qs{}{Use and LU-factorization to solve the system of linear equations
    \[
        \begin{cases}
            2x_1 = 4               \\
            -2x_1 + x_2 - x_3= -4  \\
            6x_1 + 2x_2 + x_3 = 15 \\
            -x_4 = -1
        \end{cases}
    \]
}
\solve{
The system of linear equations can be written in matrix form as
\[
    \begin{bmatrix}
        2  & 0 & 0  & 0  \\
        -2 & 1 & -1 & 0  \\
        6  & 2 & 1  & 0  \\
        0  & 0 & 0  & -1
    \end{bmatrix}
    \begin{bmatrix}
        x_1 \\ x_2 \\ x_3 \\ x_4
    \end{bmatrix}
    = \begin{bmatrix}
        4 \\ -4 \\ 15 \\ -1
    \end{bmatrix}.
\]
Let the equation be $A\bmx = \bmb$. We can factorize the coefficient matrix by
\[
    \begin{bmatrix}
        2  & 0 & 0  & 0  \\
        -2 & 1 & -1 & 0  \\
        6  & 2 & 1  & 0  \\
        0  & 0 & 0  & -1
    \end{bmatrix}
    = \begin{bmatrix}
        1  & 0 & 0 & 0 \\
        -1 & 1 & 0 & 0 \\
        3  & 0 & 1 & 0 \\
        0  & 0 & 0 & 1
    \end{bmatrix}
    \begin{bmatrix}
        2 & 0 & 0  & 0  \\
        0 & 1 & -1 & 0  \\
        0 & 0 & 3  & 0  \\
        0 & 0 & 0  & -1
    \end{bmatrix}
    = LU.
\]
Further let $U\bmx = \bmy$, where $\bmy = [y_1,\ y_2,\ y_3,\ y_4]^T$. We can
solve the system of linear equations by firstly solving $L\bmy = \bmb$ and then
solving $U\bmx = \bmy$.
\[
    \begin{cases}
        y_1 = 4          \\
        - y_1 + y_2 = -4 \\
        3y_1 + y_3 = 15  \\
        y_4 = -1
    \end{cases}
    \implies
    \begin{cases}
        y_1 = 4 \\
        y_2 = 0 \\
        y_3 = 3 \\
        y_4 = -1
    \end{cases}\\
\]
\[
    \begin{cases}
        2x_1 = 4      \\
        x_2 - x_3 = 0 \\
        3x_1 = 3      \\
        -x_4 = -1
    \end{cases}
    \implies
    \begin{cases}
        x_1 = 2 \\
        x_2 = 1 \\
        x_3 = 1 \\
        x_4 = 1
    \end{cases}.
\]
}

\end{document}